% Na podstawie \cite[s. 600]{jones_atkins}:
Jodowodór po ogrzaniu rozkłada się zgodnie z równaniem \ce{2HI -> H2 + I2}.
Niech [HI] oznacza stężenie molowe jodowodoru.
Szybkość reakcji definiuje się wtedy jako zmianę stężenia [HI] podzieloną przez przedział czasu, w którym ta zmiana nastąpiła.

% Na podstawie \cite[s. 601]{jones_atkins}:
Szybkość większości reakcji maleje w miarę zużywania się substratów.

% Na podstawie \cite[s. 603]{jones_atkins}:
Stała szybkości reakcji.
Równanie kinetyczne.
Reakcje pierwszego, drugiego, zerowego rzędu.

% Na podstawie \cite[s. 618]{jones_atkins}:
Większość reakcji przebiega szybciej w wyższej temperaturze.
Równanie Arrheniusa: $\log k = \log A - E_a/RT$.
Energia aktywacji?

Kataliza.
Katalizator.
Zatrucie katalizatora.
Enzymy.
Reakcje łańcuchowe?
