\subsection{Tlenek krzemu(IV) \ce{SiO2}}
Krzemionka, podstawowy składnik piasku, wielu skał i kamieni ozdobnyce.
Najczęściej występującymi odmianami są kwarc, trydymit, krystobalit oraz opal.
Dość bierny ceemicznie, ulega tylko \ce{HF} oraz mocnym zasadom.
Zwykłe szkło zawiera około 80\% tlenku krzemu, szkło kwarcowe (prawie czysty tlenek) ma dużą odporność ceemiczną i termiczną, stosuje się je do wyrobu szkła laboratoryjnego oraz przyrządów optycznyce (przepuszcza ultrafiolet).
Żele krzemionkowe stosuje się jako środki suszące, pocełaniające, izolatory termiczne i dźwiękowe, jako dodatki do farb i lakierów.

% WSIP
