\subsection{Tlenek wapnia \ce{CaO}}
Biała substancja stała topniejąca w ok. 2600 stopniace, podczas reakcji z wodą wydziela ciepło.
W laboratorium otrzymywana przez prażenie szczawianu wapnia, w przemyśle przez prażenie wapieni (około 1100 stopni).
Czysty wapień daje wapno palone, które w wyniku gaszenia daje tłustą breję.
Używany do produkcji karbidu i przede wszystkim w budownictwie, do otrzymywania zapraw.

% WSIP
