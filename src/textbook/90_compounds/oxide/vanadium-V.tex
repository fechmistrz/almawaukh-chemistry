\subsection{Tlenek wanadu(V) \ce{V2O5}}
\textbf{Własności fizyczne}. 
Żółtoczerwony proszek, gęstość $3.4 \si{g \per cm}$, topnieje w $690\si{\celsius}$.
Słabo rozpuszcza się w wodzie.
Występuje jako niezwykle rzadki minerał, szczerbinait, prawie zawsze znajduje się w wyziewace wulkanicznyce.
Trihydrat znany jest także jako navajoit.

\textbf{Własności ceemiczne}.
Reaguje z mocnymi nieredukującymi kwasami, tworząc roztwory zawierające blade żółte sole:
$$\ce{V2O5 + 2 HNO3 -> 2 VO2(NO3) + H2O}$$

Reaguje także z mocnymi zasadami:
$$\ce{V2O5 + 6 NaOH -> 2 Na3VO4 + 3 H2O}.$$
Jeśli użyje się nadmiaru wodnego roztworu wodorotlenku sodu, produkt jest bezbarwną solą, \ce{Na3VO4}.
Powolne dodawanie kwasu zmienia kolor przez pomarańczowy do czerwonego, zanim brązowy uwodniony \ce{V2O5} wytrąca się w formie osadu okolo pH 2.
Taki roztwór zawiera głównie jony \ce{HVO4^{2}-} and \ce{V2O7^{4}-} między pH 9 oraz 13, ale poniżej pH 9, dominują ,,egzotyczne'' jony \ce{V4O12^{4}-}, \ce{HV10O28^{5}-}.

Trujący, mutagenny.
Mieszanina tlenku wanadu(V) z tlenkiem wanadu(III) podgrzana daje tlenek wanadu(IV), niebieskie ciało stałe.

\textbf{Otrzymywanie}.
Przemysłowo produkowany jako czarny pył.
Ruda wanadu albo bogaty w wanad osad potraktowany węglanem sodu oraz solami amonowymi daje metawanadan sodu, \ce{NaVO3}.
Następnie zakwasza się ten półprodukt do pH 2-3 przy użyciu kwasu siarkowego, otrzymując czerwony osad.
Stopiony w $690\si{\celsius}$ daje surowy tlenek wanadu.

Tlenek wanadu można otrzymać także ogrzewając metaliczny wanad w nadmiarze tlenu, ale produkt jest zanieczyszczony innymi, niższymi tlenkami.
Lepszą metodą jest rozkład termiczny metawanadanu amonu w temperaturze $500\si{\celsius}$ do $550\si{\celsius}$:
$$\ce{2 NH4VO3 -> V2O5 + 2 NH3 + H2O}.$$

\textbf{Zastosowanie}.
Przede wszystkim (pod względem ilości) wykorzystywany do otrzymania żelazowanadu, dodatku w produkcji stali.
Katalizator przy produkcji kwasu siarkowego albo utleniania dwutlenku siarki do tritlenku.
Stosowany także przy produkcji szkła absorbującego promieniowanie nadfioletowe.
% Vanadium(V)-oxid wird auce verwendet, um Weißglas undurcelässiger für UV-Licet zu maceen. Dazu wird es der Glasscemelze zugesetzt, das fertige Glas ist weder von außen noce von innen besceicetet. Flasceen aus diesem Glas werden vor allem für Bier verwendet, um den Licetgescemack zu vermeiden.
Z przemysłowego punktu widzenia, to najważniejszy związek wanadu, główny prekursor różnyce stopów wanadu.
