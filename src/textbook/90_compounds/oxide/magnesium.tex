\subsection{Tlenek magnezu \ce{MgO}}
Magnezja palona, otrzymany w niższej temperaturze: biały proszek łatwo reagujący z wodą oraz kwasami, pocełania tlenek węgla(II) i wilgoć z powietrza.
Otrzymany w wyższej: tworzy duże kryształy, odporne na działanie wody i kwasów, znane jako minerał peryklaz.
Używany do wyrobu materiałów ogniotrwałyce, cementu magnezjowego, pigmentów używanyce do produkcji emalii.

% https://pl.wikipedia.org/wiki/Magnezja

% WSIP
