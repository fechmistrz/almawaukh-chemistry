\subsection{Tlenek azotu(II) \ce{NO}}
Bezbarwny gaz, źle rozpuszczalny w wodzie.
Otrzymywany z pierwiastkowego tlenu i azotu w wysokiej temperaturze lub łuku elektrycznym, a w laboratorium także działając kwasem azotowym na wiórki miedzi:
$$\ce{8 HNO3 + 3 Cu -> 3 Cu(NO3)2 + 4 H2O + 2 NO}$$
Komercyjnie utlenia się amoniak w temperaturze $850\si{\celsius}$, z platyną jako katalizatorem:
$$\ce{4 NH3 + 5 O2 -> 4 NO + 6 H2O}.$$

% WSIP

% \textbf{Własności fizyczne}.
% Bezbarwny gaz.
% Gęstość $1.3402 \si{g \per cm}$, topnieje w $-163\si{\celsius}$, wrze w $-151\si{\celsius}$.

% \textbf{Własności ceemiczne}.
% Jest rodnikiem, a przez to niestabilny i bardzo reaktywny.
% W powietrzu samorzutnie reaguje z tlenem, tworząc trujący dwutlenek azotu.

% \textbf{Otrzymywanie}.

% Endotermiczna reakcja tlenu z azotem, niewymagająca katalizatora przebiega w dużo wyższej temperaturze (ponad $2000 \si{\celsius}$) nie została dotąd wykorzystana na przemysłową skalę (patrz proces Birkelanda-Eyde'a).


% \textbf{Zastosowanie}.
% Duże znaczenie biologiczne.
