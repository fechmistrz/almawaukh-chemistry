\subsection{Tetraboran sodu \ce{Na2B4O7}}
\textbf{Własności fizyczne}.
Sproszkowany jest biały, składa się z bezbarwnych kryształów rozpuszczalnych w wodzie.
Gęstość $2.4 \si{g \per cm}$, topnieje w $743\si{\celsius}$, wrze w $1575\si{\celsius}$.
Sól dziesięciowodna topnieje i rozpada się w $75\si{\celsius}$, w przyrodzie występuje jako minerał boraks (Turcja, Kalifornia, Chile, Boliwia...).
Bardzo dobrze rozpuszczalny w glikolu, średnio w metanolu, jeszcze gorzej w acetonie.
Słabo rozpuszczalny w zimnej wodzie, ale dobrze w ciepłej.

\textbf{Własności chemiczne}.
Barwi płomień na żółto/zielono.
Podstawowa reakcja to przemiana w kwas borowy albo któryś z licznych oksoboranów, na przykład
$$\ce{Na2B4O7 * 10H2O + 2 HCl -> 4 B(OH)3 + 2 NaCl + 5 H2O}.$$

\textbf{Zastosowanie}.
Chemiczna analiza jakościowa (perły boraksowe).
Produkcja szkła, ceraminiki, wełny szklanej.
Wybielacz w detergentach, składnik pasty wybielającej zęby.
Bufor pH.
Topnik podczas spawania żelaza i stali.

\textbf{Historia}.
Odkryty na dnie tybetańskich jezior, stamtąd trafił do Półwyspu Arabskiego przez Jedwabny Szlak w VIII wieku.
