\subsection{Siarczek cynku \ce{ZnS}}
\textbf{Własności fizyczne}.
Biała krystaliczna substancja (czysty), czarna (z zanieczyszczeniami), gęstość $4.09 \si{g \per cm}$, topnieje w $1850 \si{\celsius}$ (sublimuje).
Był używany przez zespół Rutherforda do wykrywania scyntylacji: emituje światło wzbudzony promieniowaniem roentgenowskim lub strumieniem elektronów.
Dzięki temu stosowano go na początku XX wieku w zegarach z radem-226.
Występuje w dwóch formach: bardziej stabilnej, sześciennej (sfaleryt, blenda cynkowa, najpopularniejszy minerał z cynkiem w przyrodzie) oraz możliwej do otrzymania sztucznie, sześciokątnej (wurcyt).

% \textbf{Własności chemiczne}.

\textbf{Otrzymywanie}.
Produkt uboczny syntezy amoniaku z metanu, gdzie tlenek cynku usuwa resztki siarkowodorowych nieczystości z naturalnego gazu: \ce{ZnO + H2S -> ZnS + H2O}.  
W laboratorium łatwo go otrzymać zapalając mieszaninę cynku i siarki albo w reakcjach strącania.
Reakcja \ce{Zn2+ + S2- -> ZnS} jest podstawą analizy wagowej cynku.

\textbf{Zastosowanie}.
Popularny pigment, zwany czasem sachtolitem, połączony z siarczanem baru tworzy (biały) litopon o dobrym kryciu.
Zarówno sfaleryt jak i wurcyt są półprzewodnikami o szerokiej przerwie energetycznej.
Po dodaniu śladowych (kilka ppm) ilości aktywatora, ma silne właściwości fosforoscencyjne: ze srebrem daje niebieski kolor z maksimum w 450 nm, manganem -- pomarańczowo-żółty i 590 nm, zaś żelaza daje długi, zielony blask.
Drobny pył jest wydajnym fotokatalizatorem, który wytwarza gazowy wodór z wody pod wpływem naświetlania.
