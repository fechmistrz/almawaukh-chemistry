\subsection{Jodek potasu \ce{KI}}
\textbf{Własności fizyczne}.
Bezbarwne kryształy dobrze rozpuszczalne w wodzie.
Topnieje w $681 \si{\celsius}$, wrze w $1330 \si{\celsius}$.
Stare, zanieczyszczone próbki są żółte, ponieważ
$$\ce{4 KI + 2 CO2 + O2 -> 2 K2CO3 + 2 I2}$$

\textbf{Otrzymywanie}.
Przemysłowo w reakcji wodorotlenku potasu z jodem.

\textbf{Zastosowanie}.
Najważniejszy (komerycyjnie) związek jodu (37 kiloton w 1985).
Nieznaczne ilości dodaje się do soli kuchennej (sól jodowana).
Składnik niektórych środków dezynfekujących.
Wodny roztwór jodu w jodku potasu to płyn Lugola, lek na pewne schorzenia tarczycy, stosowany także podczas wypadków jądrowych.
Razem z azotanem srebra używany do produkcji jodku srebra, ważnego odczynnika w fotografii analogowej. 
Stosowany do przygotowania jodków arylowych w reakcji Sandmeyera.

\textbf{Historia}.
(opcjonalnie)



