\subsection{Siarczan baru \ce{BaSO4}}
\textbf{Własności fizyczne}.
Białe ciało stałe, trudno rozpuszczalne w wodzie.
Naturalnie występuje jako minerał baryt, główne źródło baru.
Topnieje w $1580 \si{\celsius}$, wrze w $1600 \si{\celsius}$ rozkładając się przy tym.
Gęstość $4.49 \si{g \per cm}$.

Związek ten jest otrzymywany przez mielenie barytu lub strącanie z roztworów innych soli baru kwasem siarkowym.
\textbf{Własności chemiczne}.
Charakter (kwasowy, zasadowy?), moc, pH.
Z czym reaguje?
Własności utleniajace.

\textbf{Otrzymywanie}.
Baryt jest bardzo zanieczyszczony, dlatego wygrzewa się go razem z koksem:
$$\ce{BaSO4 + 4C -> BaS + 4CO}$$
Siarczek baru jest dobrze rozpuszczalny w wodzie, można przetworzyć go na tlenek, węglan lub halogenek.
By otrzymać produkt o wysokiej czystości, do siarczku dodaje się kwasu siarkowego lub jego soli:
$$\ce{BaS + H2SO4 -> BaSO4 + H2S}$$
% Barium sulfate is reduced to barium sulfide by carbon. The accidental discovery of this conversion many centuries ago led to the discovery of the first synthetic phosphor.[4] The sulfide, unlike the sulfate, is water-soluble.

\textbf{Zastosowanie}.
Cztery piąte światowej produkcji (głównie z barytu) zużywa się jako płuczkę wiertniczą w szybach naftowych.
Środek kontrastowy przy prześwietleniach przewodu pokarmowego.
Z syntetycznego otrzymuje się farbę (biel barowa/barytowa).
Składnik zielonych materiałów pirotechnicznych.
