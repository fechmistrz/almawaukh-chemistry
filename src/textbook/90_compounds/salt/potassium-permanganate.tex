\subsection{Nadmanganian potasu \ce{KMnO4}}
\textbf{Własności fizyczne}. 
Ciemnofioletowe kryształy o metalicznym połysku, gęstość $2.7 \si{g \per cm}$, topnieje w $240\si{\celsius}$ (rozkłada się z wydzieleniem tlenu).
Niehigroskopijny, umiarkowanie rozpuszczalny w wodzie ze zmianą koloru w intensywny różowy lub fioletowy.

\textbf{Własności chemiczne}. 
Silny utleniacz (alkohole pierwszorzędowe od razu do soli kwasów karboksylowych), jednocześnie nie jest łatwopalny.
Gwałtownie reaguje ze sproszkowanymi metalami.
W roztworach kwaśnych redukuje się do bezbarwnego lub bladoróżowego roztworu manganu(II), w neutralnych do brązowego lub brunatnego osadu: tlenku manganu(IV), braunsztynu; w zasadowych do zielonego manganianu.
Wykrywa jony redukujące (tiocyjaniany, bromki, azotyny).
Miareczkowanie.
Stosuje się go podczas syntezy kwasu askorbinowego, izonikotynowego, chloramfenikolu, sacharyny.
Historycznie wykrywał podwójne i potrójne wiązania węglowe (odczynnika Baeyera).
Kontakt ze skórą prowadzi do długo utrzymujących się różowych plam.

\textbf{Otrzymywanie}.
Przemysłowo otrzymuje się go z dwutlenku manganu (występującego także jako minerał, piroluzyt).
Stopiony z wodorotlenkiem potasu i podgrzany w powietrzu lub innym źródle tlenu (azotanie lub chloranie potasu) jest źródłem soli: $$\ce{2 MnO2 + 4 KOH + O2 -> 2 K2MnO4 + 2 H2O},$$ która podczas elektrolitycznego utleniania w alkalicznym ośrodku daje nadmanganian: $$\ce{2 K2MnO4 + 2 H2O -> 2 KMnO4 + 2 KOH + H2}$$
Inne metody (nieprzemysłowe): utlenianie manganianu potasu przez chlor lub dysproporcjonowanie w warunkach kwasowych; albo traktowanie roztworu jonów manganowych przez silne utleniacze takie jak tlenek ołowiu, bizmutan sodu albo jon nadsiarczanowy.

\textbf{Zastosowanie}.
Bakterio- i grzybobójczy.
Roztwór o stężeniu 0,05\% przemywa skórę, rany i owrzodzenia; roztworem 0,025\% można płukać jamę ustną.
Usuwa żelazo i kwas siarkowy ze studni (,,manganese greensand filter'').
Prekursor narkotykowy kategorii 2: utlenianie pasty kokainowej oczyszcza ją i stabilizuje.
Wymieszany z glicerolem ulega samozapłonowi (rozpalanie ognia w trudnych warunkach), inicjuje termitu.

\textbf{Historia}.
W 1659 roku J. Glauber stopił mieszaninę piroluzytu z węglanem potasu.
Otrzymana substancja rozpuszczona w wodzie miała zielony kolor, który powoli przechodził w purpurę i ostatecznie czerwień.
Londyński chemik H. Condy dwieście lat później stopił piroluzyt z wodorotlenkiem sodu i dostał roztwór o właściwościach dezynfekujących.
Zastąpienie sodu potasem dawało stabilniejszy roztwór.
