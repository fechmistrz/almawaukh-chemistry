\subsection{Siarczek rtęci(II) \ce{HgS}}
\textbf{Własności fizyczne}.
Występuje w przyrodzie jako czerwony minerał cynober (odmiana $\alpha$), rzadziej także jako czarny metacynober (odmiana $\beta$).
Praktycznie nierozpuszczalny.
Rozkłada się podczas topnienia w $580 \si{\celsius}$, gęstość $8.10\si{g \per cm}$.

\textbf{Zastosowanie}.
Odmianę $\alpha$ stosuje się jako pigment vermilion (chińska czerwień).
Z czasem ciemnieje ze względu na powstawanie związków rtęci i chloru (kordieryt, kalomel) oraz gipsu (siarczanu wapnia: w Pompejach? wszędzie?).
Najpopularniejsze źródło rtęci wśród rud.
