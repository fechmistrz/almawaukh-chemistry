\subsection{Chloran(V) potasu \ce{KClO3}}
\textbf{Własności fizyczne}.
Biała krystaliczna substancja, gęstość $2.32 \si{g \per cm}$, topnieje w $356 \si{\celsius}$, wrze w $400 \si{\celsius}$ (rozpada się).
Najpopularniejszy chloran w przemyśle.

\textbf{Otrzymywanie}.
Na skalę przemysłową w procesie Liebiga: przepuszczanie chloru przez gorący wodorotlenek wapnia i dodanie chlorku potasu:
$$\ce{6 Ca(OH)2 + 6 Cl2 -> Ca(ClO3)2 + 5 CaCl2 + 6 H2O}$$ 
$$\ce{Ca(ClO3)2 + 2 KCl -> 2 KClO3 + CaCl2}$$
% The electrolysis of KCl in aqueous solution is also used sometimes, in which elemental chlorine formed at the anode react with KOH in situ. The low solubility of KClO3 in water causes the salt to conveniently isolate itself from the reaction mixture by simply precipitating out of solution.

\textbf{Zastosowanie}.
Był głównym składnikiem kapiszonów: materiały pędne oparte na chloranach są wydajniejsze od tradycyjnego prochu czarnego i mniej podatne na działanie wody, ale są też niestabilne w obecności siarki lub fosforu i dużo droższe.
Często łączony z piorunianem srebra w strzelających diabełkach.
% Potassium chlorate is often used in high school and college laboratories to generate oxygen gas.[citation needed] It is a far cheaper source than a pressurized or cryogenic oxygen tank. Potassium chlorate readily decomposes if heated while in contact with a catalyst, typically manganese(IV) dioxide (MnO2). Thus, it may be simply placed in a test tube and heated over a burner. If the test tube is equipped with a one-holed stopper and hose, warm oxygen can be drawn off. The reaction is as follows:
% 2 KClO3(s) → 3 O2(g) + 2 KCl(s)
% Heating it in the absence of a catalyst converts it into potassium perchlorate:[9]
% 4 KClO3 → 3 KClO4 + KCl
% With further heating, potassium perchlorate decomposes to potassium chloride and oxygen:
% KClO4 → KCl + 2 O2
% The safe performance of this reaction requires very pure reagents and careful temperature control. Molten potassium chlorate is an extremely powerful oxidizer and spontaneously reacts with many common materials such as sugar. Explosions have resulted from liquid chlorates spattering into the latex or PVC tubes of oxygen generators, as well as from contact between chlorates and hydrocarbon sealing greases. Impurities in potassium chlorate itself can also cause problems. When working with a new batch of potassium chlorate, it is advisable to take a small sample (~1 gram) and heat it strongly on an open glass plate. Contamination may cause this small quantity to explode, indicating that the chlorate should be discarded.
Rebelianci z Afganistanu używają go jako kluczowego składnika improwizowanych ładunków wybuchowych po tym, jak ograniczono dostęp do saletry amonowej.
Utleniacz w granatach dymnych, od 2005 kartridż z chloranem, laktozą i~kalafonią wytwarza biały dym oznajmujący wybranie papieża.
Stosowany także do produkcji zapałek.
% Potassium chlorate is used in chemical oxygen generators (also called chlorate candles or oxygen candles), employed as oxygen-supply systems of e.g. aircraft, space stations, and submarines, and has been responsible for at least one plane crash. A fire on the space station Mir was also traced to this substance. The decomposition of potassium chlorate was also used to provide the oxygen supply for limelights.
Pestycyd, w Finlandii sprzedawany pod nazwą Fegabit.
