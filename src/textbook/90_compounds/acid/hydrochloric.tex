\subsection{Kwas solny \ce{HCl}}
\textbf{Własności fizyczne}.
Czysty kwas jest bezbarwny, techniczny ma żółtawe zabarwienie, gdyż jest zanieczyszczony jonami żelaza.
Maksymalne stężenie wynosi 45\%, ze stężonego kwasu ulatnia się gazowy chlorowodór, który reagując z wilgocią z powietrza tworzy mgłę; stąd kwas określa się jako dymiący.
Poniżej 30\% już nie jest dymiący, ale wciąż lotny.
Ma wtedy gęstość $1.149 \si{g \per cm^3}$, topnieje w $-52\si{\celsius}$, wrze w $90\si{\celsius}$.
Wyraźny, gryzący zapach.

\textbf{Własności chemiczne}.
Jeden z najmocniejszych kwasów nieorganicznych.
Nie ma właściwości utleniających.

\textbf{Otrzymywanie}.
W XV wieku Basilius Valentinus otrzymał go z soli kamiennej oraz siarczanu żelaza(II).

% TODO: During the Industrial Revolution in Europe, demand for alkaline substances increased. A new industrial process developed by Nicolas Leblanc of Issoudun, France enabled cheap large-scale production of sodium carbonate (soda ash). In this Leblanc process, common salt is converted to soda ash, using sulfuric acid, limestone, and coal, releasing hydrogen chloride as a by-product. Until the British Alkali Act 1863 and similar legislation in other countries, the excess HCl was vented into the air. After the passage of the act, soda ash producers were obliged to absorb the waste gas in water, producing hydrochloric acid on an industrial scale.[13][25]

% TODO: In the 20th century, the Leblanc process was effectively replaced by the Solvay process without a hydrochloric acid by-product. Since hydrochloric acid was already fully settled as an important chemical in numerous applications, the commercial interest initiated other production methods, some of which are still used today. After the year 2000, hydrochloric acid is mostly made by absorbing by-product hydrogen chloride from industrial organic compounds production.[13][25][26]

\textbf{Zastosowanie}.
Jest jednym z najważniejszych kwasów w przemyśle (m.in. przemysł włókienniczy, tworzyw sztucznych, farmaceutyczny, garbarstwo, cukrownictwo, produkcja żelatyny, barwników, ekstrakcja rud).
Trawienie stali: usuwa rdzę lub tlenki żelaza przed innymi procesami, takimi jak tłoczenie, walcowanie, cynkowanie. Używa się do tego technicznego kwasu o stężeniu 18\%.
Produkcja związków organicznych (chlorek winylu, dichlorek etylenu) dla PVC, bisfenolu A, poliwęglanów, kwasu askorbinowego i innych.
Produkcja związków nieorganicznych: chlorku wapnia (zapobiega oblodzeniom na drogach), chlorku niklu (galwanostegia), chlorku cynku (cynkowanie, produkcja baterii).
Prekursor narkotykowy 3 kategorii.
