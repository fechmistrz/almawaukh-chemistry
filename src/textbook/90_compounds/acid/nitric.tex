\subsection{Kwas azotowy(V) \ce{HNO3}}
\textbf{Własności fizyczne}.
Dymiąca bezbarwna ciecz, topnieje w $-41\si{\celsius}$, wrze w $86\si{\celsius}$.
Pod wpływem światła lub ogrzewania rozkłada się na tlenki o żółtej barwie.
Tworzy z wodą azeotrop.

\textbf{Własności chemiczne}.
Jego pary są toksyczne.
Czysty kwas jest wybuchowy.
Reaguje z większością metali, na jego działanie odporne są złoto, platyna, rod, iryd, tantal oraz niektóre metale nieszlachetne ulegające pasywacji (żelazo, glin, itd.).
W reakcji stężonego kwasu z białkami powstają żółto zabarwione produkty (reakcja ksantoproteinowa, stosowana w analizie do wykrywania białek).
Stężony działa żrąco na tkaniny i skórę.

\textbf{Otrzymywanie}.
Metoda katalityczna: utlenianie amoniaku do tlenku azotu(II), który jest dalej utleniany  do tlenku azotu(IV) tlenem z powietrza, a następnie absorbowany w wodzie.

\textbf{Zastosowanie}.
Produkcja nawozów, włókien sztucznych (nylony), materiałów wybuchowych (nitrogliceryna), barwników, lekarstw, azotanu(V) srebra(I) dla przemysłu fotograficznego, jako  utleniacz w rakietowych materiałach pędnych.
