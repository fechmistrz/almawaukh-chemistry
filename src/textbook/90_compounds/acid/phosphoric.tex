\subsection{Kwas ortofosforowy \ce{H_3PO_4}}
\textbf{Własności fizyczne}.
Bezbarwne ciało stałe (kryształy), gęstość $2.03 \si{g \per cm}$, topnieje w $42\si{\celsius}$, wrze w $407\si{\celsius}$.
Najczęściej występuje jako lepki, nielotny, bezbarwny i bezwonny roztwór wodny 85\%.
Higroskopijny.
% but still pourable!
Rozpuszczalny także w etanolu.

\textbf{Własności chemiczne}.
Słaby kwas trójprotonowy.
Roztwór wodny pomimo to drażni skórę oraz uszkadza oczy.
Gwałtownie reaguje z zasadami, polimeryzuje się pod wpływem azozwiązków i epoksydów.
Rozkłada się podczas kontaktu z alkoholami, aldehydami, cyjankami, ketonami, fenolami, estrami, siarczkami, wydzielając przy tym toksyczne opary.

\textbf{Otrzymywanie}.
W przemyśle stosuje się dwie metody.
Proces mokry polega na potraktowaniu kwasem siarkowym minerałów zawierających fosforan, np. hydroksyapatytu: $$\ce{Ca5(PO4)3OH + 5 H2SO4 -> 3 H3PO4 + 5 CaSO4 v + H2O}$$ albo apatytu, fosforytu, czasami także zmielonych kości zwierząt:
$$\ce{Ca3(PO4)2 + 3 H2SO4 -> 2 H3PO4 + 3 CaSO4}$$
Wydzielający się w tej reakcji siarczan wapnia można łatwo oddzielić od kwasu fosforowego przez filtrowanie, gdyż nie rozpuszcza się on w wodzie.
Alternatywnym źródłem jest fluoroapatyt, gdzie powstaje nierozpuszczalny fluorek \ce{Na2SiF6}.

\textbf{Zastosowanie}. Głównie (90\%) do produkcji nawozów sztucznych takich jak superfosfat podwójny. W przemyśle spożywczym jako regulator kwasowości E338.
% Stosowany jest też do wytwarzania fosforanowych powłok ochronnych na metalach, do wytwarzania wielu środków farmaceutycznych, oczyszczania soków w cukrownictwie, odkamieniania armatury w ciepłownictwie, jako płyn do lutowania, w stomatologii, do wyrobu kitów porcelanowych, w lecznictwie i laboratoriach analitycznych. Jest także składnikiem fosolu – odrdzewiacza do stali.
