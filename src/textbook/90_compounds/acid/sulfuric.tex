\subsection{Kwas siarkowy(VI) \ce{H2SO4}}
\textbf{Własności fizyczne}.
Bezbarwna, oleista ciecz, topnieje w $10\si{\celsius}$, wrze w $320\si{\celsius}$ (z częściowym rozkładem).
Miesza się z wodą w każdym stosunku, podczas czego może wydzielić się tak dużo ciepła, że temperatura mieszaniny podniesie się o ponad $100\si{\celsius}$.

\textbf{Własności chemiczne}.
Gorący, stężony kwas jest silnym utleniaczem, reaguje z miedzią, srebrem, rtęcią, utlenia też niemetale (redukując się przy tym do tlenku siarki(IV)).
Rozcieńczony nie ma właściwości utleniających.
Stężony jest silnym środkiem odwadniającym, zwęgla cukier i drewno, powoduje oparzenia i trudno gojące się rany.

\textbf{Otrzymywanie}.
Metoda kontaktowa: katalityczne utlenienie tlenku siarki(IV).

\textbf{Zastosowanie}.
Produkcja innych kwasów, nawozów sztucznych (superfosfat), osuszanie gazów, w garbarstwie i przemyśle spożywczym, włókienniczym i papierniczym, jako elektrolit w akumulatorach ołowiowych.
