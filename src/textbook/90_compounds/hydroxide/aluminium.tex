\subsection{Wodorotlenek glinu \ce{Al(OH)3}}
\textbf{Własności fizyczne}.
Biały bezwonny, amorficzny proszek o gęstości $2.42 \si{g \per cm^3}$.
Temperatura topnienia $300 \si{\celsius}$.
Nie rozpuszcza się w etanolu ani w wodzie.
Naturalnie występuje jako minerał w czterech odmianach polimorficznych: gibbsyt, bajeryt, doyleite, nordstrandyt.

\textbf{Własności chemiczne}.
Amfoteryczny (kwasowo-zasadowy), pH powyżej 7.
Niepalny.

\textbf{Otrzymywanie}.
Przemysłowo niemal w całości otrzymuje się go w procesie Bayera (1887 r.), gdzie boksyt rozpuszcza się w roztworze wodorotlenku sodu do około $270 \si{\celsius}$.
Odpady (czerwony szlam) usuwa się, a następnie strąca wodorotlenek z pozostałego roztworu glinianu sodu.

\textbf{Zastosowanie}.
Środek opóźniający palenie: rozpada się w około $180 \si{\celsius}$, pochłaniając część ciepła i oddając parę wodną.
Środek tłumiący dym, głównie w poliestrach, akrylach, EVA, PVC, żywicach epoksydowych. 
W farmacji: środek na nadkwaśność żołądka (zobojętnia kwas, nierozpuszczalny, nie zwiększa pH powyżej 7), substancja ścierająca i polerująca przy produkcji past do zębów.
Produkcja papieru, mydła, kosmetyków.
