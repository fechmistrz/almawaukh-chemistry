\subsection{Wodorotlenek wapnia \ce{Ca(OH)2}}
% Unprotected exposure to Ca(OH)2 can cause severe skin irritation, chemical burns, blindness or lung damage or rashes [5]
% Limewater is the common name for a saturated solution of calcium hydroxide.
% One significant application of calcium hydroxide is as a flocculant, in water and sewage treatment. It forms a fluffy charged solid that aids in the removal of smaller particles from water, resulting in a clearer product.
% Another large application is in the paper industry, where it is an intermediate in the reaction in the production of sodium hydroxide. This conversion is part of the causticizing step in the Kraft process for making pulp.[8] In the causticizing operation, burned lime is added to green liquor, which is a solution primarily of sodium carbonate and sodium sulfate produced by dissolving smelt, which is the molten form of these chemicals from the recovery furnace.
% Unprotected exposure to Ca(OH)2 can cause severe skin irritation, chemical burns, blindness or lung damage or rashes [5]
\textbf{Informacje ogólne}
Jako minerał, portlandyt, występuje stosunkowo rzadko: w skałach wulkanicznych oraz przeobrażonych.

\textbf{Własności fizyczne}.
Bezwonne, bezbarwne kryształy lub biały proszek.
Rozkłada się podczas topnienia w $580 \si{\celsius}$ na tlenek wapnia i wodę, gęstość $2.211 \si{g \per cm}$.
Słabo rozpuszczalny w wodzie.

\textbf{Własności chemiczne}.
Jego wodny roztwór nazywa się wodą wapienną, jest dość mocną zasadą (pH ok. 12.4) o żrącym działaniu, która w obecności dwutlenku węgla mętnieje wskutek wytrącania się węglanu wapnia.
(Roztwór) atakuje niektóre metale (glin), chroniąc jednocześnie inne przed korozją (żelazo, stal) przez pasywację.
Stosowany przy produkcji amoniaku: $$\ce{Ca(OH)2 + 2NH4Cl ->  2NH3 + CaCl2 + 2H2O}$$

\textbf{Otrzymywanie}.
W laboratorium można wymieszać wodne roztwory chlorku wapnia oraz wodorotlenku sodu.
Nazwa wapno gaszone (albo lasowane) pochodzi od przemysłowej metody otrzymywania: $$\ce{CaO + H2O -> Ca(OH)2}$$

\textbf{Zastosowanie}.
W budownictwie jako spoiwo zapraw murarskiej i tynkarskiej, jako baza i lepiszcze farb malarskich o właściwościach dezynfekujących.
W cukrownictwie do oczyszczania soku buraczanego, jako zmiękczacz wody.
W energetyce do odsiarczania spalin.
Składnik cementu stomatologicznego oraz konserwant żywności (E526).

