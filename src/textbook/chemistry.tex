\documentclass{createspace}

% https://tex.stackexchange.com/questions/69901/how-to-typeset-greek-letters
\newcommand{\textgreek}[1]{\begingroup\fontencoding{LGR}\selectfont#1\endgroup}

\usepackage{xcolor}
\usepackage{tcolorbox}
\newtcolorbox{etymology_box}{colback=blue!5!white, colframe=blue!75!black,fonttitle=\bfseries,title=Etymologia}

\usepackage{siunitx}
\usepackage[version=4]{mhchem}

\newcommand{\important}[1]{{\color{blue!70!white}\textbf{#1}}}

\usepackage{etoolbox} % https://tex.stackexchange.com/questions/492461/showidx-gives-undefined-control-sequence-error
% forced by -output-directory option in makefile
% see https://tex.stackexchange.com/a/564296
\makeatletter
\patchcmd\imki@putindex
  {\imki@exec{\imki@program \imki@options #1.idx}}
  {\imki@exec{cd ../../build/textbook;\imki@program\imki@options#1.idx}}
  {\message{Patch succeeded in imki@putindex}}
  {\errmessage{Patch failed in imki@putindex}}
\makeatother

\makeindex[title=Skorowidz]
\makeindex[name=persons,title=Indeks osób]

\author{Aallotar Järvinen}
\title{Chemiczne notatki}

\begin{document}

% strona pierwsza

\thispagestyle{empty}
{\noindent\fontsize{18pt}{18pt}\selectfont Księgozbiór alchemiczny, tom 97}

\noindent\makebox[\linewidth]{\rule{\textwidth}{1pt}}

\newpage

% koniec strony pierwszej



% strona druga

\thispagestyle{empty}
\phantom{nothing}
\newpage

% koniec strony drugiej



% strona trzecia

\thispagestyle{empty}
{\noindent\fontsize{18pt}{18pt}\selectfont Aallotar Järvinen}

\noindent\makebox[\linewidth]{\rule{\textwidth}{1pt}}

\vspace{10mm}

{\noindent\fontsize{24pt}{24pt}\selectfont \textbf{Chemiczne notatki\\(takie tam)}}
\vspace{10mm}

{\noindent\fontsize{14pt}{14pt}\selectfont Wydanie siódme niepoprawione}

\newpage

% koniec strony trzeciej



% strona czwarta

\thispagestyle{empty}
\begin{figure}[H]
\begin{minipage}[b]{.48\linewidth}
{\noindent Prof. Aallotar Järvinen\\
Helsingin yliopisto\\
A. I. Virtasen aukio 1\\
00560 Helsinki, Finlandia}
\end{minipage}
%\begin{minipage}[b]{.48\linewidth}
%{\noindent Imię Nazwisko\\
%Szkoła\\
%Ulica\\
%Miasto, Kraj}
%\end{minipage}
\end{figure}

{\noindent \textbf{Tytuł oryginału}\\Kemiaa kaikkialla}
\vspace{5mm}

{\noindent \textbf{Z fińskiego tłumaczyła}\\Johanna Virtanen} 
\vspace{5mm}

{\noindent \textbf{Okładkę zaprojektował}\\Ilmari Heikkinen}
\vspace{5mm}

{\noindent \textbf{Zredagował}\\Juhani Koskinen}
\vspace{5mm}

{\noindent \textbf{Zredagowała technicznie}\\Marjatta Korhonen}
\vspace{5mm}

{\noindent \textbf{Złożyli i połamali}\\???, Oulu}
\vspace{5mm}

{\noindent \textbf{Korekty dokonali}\\Liisa Mäkelä, Tapio Paikkala}

\vfill

{\noindent Copyleft by Antykwariat Czarnoksięski, Gorzów Wielkopolski 2022.
Książka, a także każda jej część, mogą być przedrukowywane oraz w jakikolwiek inny sposób reprodukowane czy powielane mechanicznie, fotooptycznie, zapisywane elektronicznie lub magnetycznie, oraz odczytywane w środkach publicznego przekazu bez pisemnej zgody wydawcy.}

\vspace{5mm}

{\noindent Przygotowano w systemie \TeX, wydrukowano na siarczystym papierze.}

% koniec strony czwartej



% strona piąta

\chapter*{Przedmowa}
Blablabla.
Z takim zakresem i ujęciem materiału pozycja jest unikalna nie tylko w fińskiej, ale i w światowej literaturze chemicznej.

Serdecznie dziękuję A. B. oraz Ł. P. bez których ten tekst nigdy by nie powstał.\\${}$

\begin{flushright}
A. Järvinen,\\Oulu, 22 lutego 2022
\end{flushright}

\section*{Przedmowa do wydania szóstego}
Blablabla.
Żywię nadzieję, że korzystanie z~książki będzie równie (jeśli nie bardziej) przyjemne, co w~przypadku szóstego jej wydania.

\begin{flushright}
A. Järvinen,\\Espoo, 15 marca 2019
\end{flushright}

\section*{Przedmowa do wydania piątego}
Do napisania...

\begin{flushright}
A. Järvinen,\\Espoo, 11 sierpnia 2017
\end{flushright}

% koniec strony piątej


\tableofcontents

\chapter{Budowa atomu i cząsteczki. Układ okresowy pierwiastków}
% Jones, Atkins: 1, 7, 9
% masę elektronu wyznaczył Millikan

\section{Budowa atomu}
% Jones, Atkins: 1, 7, 9
Nad teorią dotyczącą budowy materii pracowano już w starożytności.
Greccy filozofowie Leucyp z Miletu oraz jego uczeń Demokryt z Abdery sądzili, że otaczające ich substancje są zbudowane z małych niepodzielnych drobin (z greckiego \emph{atomos} -- niepodzielny).
\index[persons]{Leucyp z Miletu}%
\index[persons]{Demokryt z Abdery}%
Ich pomysły rozwijali potem Epikur oraz Lukrecjusz, ale niepoparte doświadczeniami pozostały zapomniane aż do wczesnego średniowiecza.
\index[persons]{Epikur}%
\index[persons]{Lukrecjusz}%
Atomizm kojarzono z epikureizmem, który był sprzeczny z nauczaniem chrześcijańskim, więc większość europejskich filozofów nie akceptowała istnienia atomów.
Francuski duchowny Pierre Gassendi zaproponował pewne zmiany w tych koncepcjach i jako pierwszy użył terminu ,,cząsteczka''.
% molecule
\index[persons]{Gassendi, Pierre}%
Obrońcami atomizmu byli angielski chemik Robert Boyle oraz angielski fizyk Izaak Newton tak, że pod koniec XVII wieku pogląd ten był już akceptowany przez część społeczności naukowej.
\index[persons]{Boyle, Robert}%
\index[persons]{Newton, Izaak}%

Pod koniec XVIII wieku znano już prawo zachowania masy:
\begin{theorem}[prawo zachowania masy]
	Suma mas spoczynkowych ciał biorących udział w procesie nie zmienia się.
\end{theorem}
(sformułowane niezależnie od siebie przez rosyjskiego chemika Michaiła Łomonosowa w 1756 r. i francuskiego chemika Antoine de Lavoisier w 1785 r.) oraz prawo stosunków stałych albo stałości składu:
\index[persons]{Łomonosow, Michaił}%
\index[persons]{Lavoisier, Antoine}%
\begin{theorem}[prawo stosunków stałych]
	Pierwiastki chemiczne łączą się ze sobą w określony związek chemiczny zawsze w tym samym stosunku wagowym.
\end{theorem}
(podane po raz pierwszy około 1799 r. przez francuskiego chemika Josepha Prousta, potwierdzone przez belgijskiego chemika Jeana Stasa).
\index[persons]{Proust, Joseph}%
\index[persons]{Stas, Jean}%
John Dalton zaproponował jeszcze prawo stosunków wielokrotnych:
\index[persons]{Dalton, John}%

\begin{tcolorbox}[title={Do zrobienia poźniej}]
Oprócz związków spełniających prawo Prousta (tzw. daltonidy) istnieją związki nie spełniające go (tzw. bertolidy).
\end{tcolorbox}

\begin{theorem}[prawo stosunków wielokrotnych]
Jeżeli dwa pierwiastki A i B tworzą ze sobą więcej niż jeden związek, to masy pierwiastka A przypadające na taką samą masę pierwiastka B mają się do siebie jak niewielkie liczby całkowite.
\end{theorem}

Ten sam Dalton ogłosił w 1803 r. założenia hipotezy atomistycznej, które tłumaczyły doświadczalne prawa chemiczne.
Założeniami tymi były:
\begin{compactitem}
	\item materia zbudowana jest z atomów, 
	\item atomy jednego pierwiastka są identyczne i mają kształt kuli,
	\item atomy są niezmienne i niepodzelne,
	\item związki chemiczne powstają przez łączenie się atomów.
\end{compactitem}

% \subsection{Avogadro}
% Usterkę w teorii Daltona poprawił Amedeo Avogadro w 1811 roku.
% Zasugerował on, że równe objętości dowolnych dwóch gazów w tej samej temperaturze i pod tym samym ciśnieniem zawierają równe ilości cząsteczek tych gazów.
% Prawo to pozwoliło wywnioskować, że wiele gazów ma dwuatomowe cząsteczki. % , na przykład: dwa litry wodoru reagują z litrem tlenu i dają dwa litry pary wodnej

\textbf{Ruchy Browna}.
W 1827 r. szkocki botanik Robert Brown zauważył bezładne ruchy po zawikłanych torach pyłków kwiatowych unoszących się na wodzie.
Ruch ten nie słabnie w czasie i staje się bardziej intensywny wraz ze wzrostem temperatury wody.
\index[persons]{Brown, Robert}%
W 1881 r. polski fizyk Łukasz Bodaszewski zaobserwował i opisał podobne zjawisko w gazach.
\index[persons]{Bodaszewski, Łukasz}%
Pierwsze teoretyczne wyjaśnienie ruchów Browna podali niezależnie Albert Einstein w 1905 r. i Marian Smoluchowski w 1906 roku: cząsteczki wody nieustannie uderzają o pyłki i powodują ich ruch.
(Matematyczny opis ruchów Browna był jedną z przyczyn, dla których powstała teoria procesów stochastycznych).
\index[persons]{Einstein, Albert}%
\index[persons]{Smoluchowski, Marian}%
Wyniki Einsteina potwierdził w 1908 r. na drodze doświadczalnej francuski fizyk Jean Perrin.
\index[persons]{Perrin, Jean}%

\textbf{Odkrycie elektronu, model ciasta z rodzynkami Thomsona}.
Atomy uważano za najmniejsze porcje materii aż do 1897 r., kiedy Joseph Thomson, John Townsend, Harold Wilson odkryli ,,korpuskuły'' podczas swoich doświadczeń nad promieniowaniem katodowym w rurze Crookesa-Hittorfa; ostatecznie przyjęła się nazwa zaproponowana w 1891 r. przez irlandzkiego fizyka George'a Stoneya: ,,\emph{elektrony}''.
\index[persons]{Thomson, Joseph}%
\index[persons]{Townsen, John}%
\index[persons]{Wilson, Harold}%
\index{rura Crookesa-Hittorfa}%
% https://en.wikipedia.org/wiki/Crookes_tube
Thomson rozstrzygnął przez zastosowanie komory Wilsona, że masa elektronu jest 1800 razy mniejsza niż masa cząsteczki wodoru.
\index{komora Wilsona}%
Aby wytłumaczyć, czemu ładunek całego atomu jest zerowy, podał \emph{model ciasta z rodzynkami}: elektrony to punktowe ujemne ładunki w ciągłym przestrzennie ładunku dodatnim.

\textbf{Odkrycie jądra, model planetarny Rutherforda}.
Hans Geiger i Ernest Marsden pod nadzorem Ernesta Rutherforda w latach 1908-1913 bombardowali cząstkami $\alpha$ cienką folię ze złota .
\index[persons]{Geiger, Hans}%
\index[persons]{Marsden, Ernest}%
\index[persons]{Rutherford, Ernest}
Gdyby model Thomsona był prawdziwy, cząstki $\alpha$ powinny przenikać przez folię, ewentualnie z lekkim odchyleniem toru.
Ale uczeni zauważyli duży rozrzut cząstek na fluoresencyjnym ekranie wokół folii, niektóre z nich odskakiwały w tył.
Doszli do wniosku, że niemal cała masa i dodatni ładunek skupione są w jądrze atomowym o bardzo małych rozmiarach; jednocześnie obalając model Thomsona.
Według Rutherforda pomysł planetarnego modelu pochodził od Hantaro Nagaoki (,,pierścienie wokół Saturna'').
\index[persons]{Hantaro Nagaoka}

\textbf{Model Bohra}.
Planetarny model atomu miał dwie poważne wady: po pierwsze elektrony powinny emitować fale elektromagnetyczne zgodnie z wzorem\footnote{$P = q^2a^2/(6\pi\varepsilon_0 c^3$} Larmora, stale tracić energię i wirować w kierunku jądra, zderzając się z nim w czasie około $10^{-6} s$.
Po drugie, model ten nie tłumaczył wysokich pików na widmach emisyjnych i absorbcyjnych atomów.
% Kwantowy model A. Haasa z 1910 r. oraz J. Nicholsona z 1912 roku.
By wyjaśnić te sprzeczności duński fizyk Niels Bohr zaproponował w 1913 r. teorię budowy atomu, gdzie elektron krąży wokół jądra atomowego po orbitach stacjonarnych o ściśle określonej energii.
\index[persons]{Bohr, Niels}%
Orbitalny moment pędu jest skwantowany (jest całkowitą wielokrotnością h kreślonego).
Elektron przechodząc między orbitami emituje lub pochłania foton i nie może spaść na jądro atomowe.
Model Bohra nie był doskonały: przewidywał jedynie linie widmowe wodoru, ale nie radził sobie z atomami o większej liczbie elektronów.
Wraz z rozwojem spektrografii zaobserwowano dodatkowe linie widmowe wodoru, których ten model nie tłumaczył.
W 1916 r. Arnold Sommerfeld dokonał poprawki modelu przez dodanie eliptycznych orbit, ale to uczyniło go trudnym w użyciu i nadal nie tłumaczyło innych pierwiastków.
\index[persons]{Sommerfeld, Arnold}%

\textbf{Odkrycie izotopów}.
W 1913 r. brytyjski chemik Frederick Soddy odkrył, że radioaktywne pierwiastki mogą mieć wiele odmian o różnych masach atomowych, ale tych samych własnościach chemicznych.
Zgodnie z sugestią Margaret Todd nazwał te odmiany \emph{izotopami}.
\index[persons]{Soddy, Frederick}%
\index[persons]{Todd, Margaret}%
Thomson odkrył z Francisem Astonem w 1913 r. na przykładzie neonu, że nieradioaktywne pierwiastki też mogą mieć wiele izotopów.
\index[persons]{Thomson, Joseph}%
\index[persons]{Aston, Francis}%

% Soddy przewidział także końcowe produkty rozpadu szeregu uranowego i torowego (są to izotopy ołowiu o masach atomowych 206 i 208).
% W 1917 odkrył izotop protaktynu 231Pa (niezależnie od Soddy’ego odkryli go także O. Hahn i L. Meitner).
% Soddy i Rutherford wprowadzili pojęcie „czasu połowicznego rozpadu” pierwiastków promieniotwórczych.

\textbf{Odkrycie protonu}.
% TODO: The concept of a hydrogen-like particle as a constituent of other atoms was developed over a long period. As early as 1815, William Prout proposed that all atoms are composed of hydrogen atoms (which he called "protyles"), based on a simplistic interpretation of early values of atomic weights (see Prout's hypothesis), which was disproved when more accurate values were measured.[16]: 39–42 
% In 1886, Eugen Goldstein discovered canal rays (also known as anode rays) and showed that they were positively charged particles (ions) produced from gases. However, since particles from different gases had different values of charge-to-mass ratio (e/m), they could not be identified with a single particle, unlike the negative electrons discovered by J. J. Thomson. Wilhelm Wien in 1898 identified the hydrogen ion as the particle with the highest charge-to-mass ratio in ionized gases.[17]
% Following the discovery of the atomic nucleus by Ernest Rutherford in 1911, Antonius van den Broek proposed that the place of each element in the periodic table (its atomic number) is equal to its nuclear charge. This was confirmed experimentally by Henry Moseley in 1913 using X-ray spectra.


W 1917 r. Rutherford pokazał, że jądro wodoru jest obecne w jądrach innych pierwiastków.
Zauważył najpierw, że kiedy strzelał cząsteczkami $\alpha$ w powietrzu (złożonym głównie z azotu), jednym z produktów były jądra wodoru, a potem że to samo doświadczenie przeprowadzone w czystym azocie daje mocniejszy efekt.
W 1919 r. założył, że promieniowanie $\alpha$ wybija proton z azotu, zamieniając go w węgiel, ale po 1925 r. zrozumiał, że równocześnie cząstka $\alpha$ była pochłaniania, zaś jądra wodoru emitowane, dając atomy tlenu, nie węgla: \ce{^{14}N + {\alpha} -> ^{17}O + p}.

Rutherford wiedział, że dodatni ładunek dowolnego atomu jest całkowitą krotnością jąder wodoru, ale przypuszczał więcej: że jądro wodoru jest pojedynczą cząstką, podstawowym składnikiem wszystkich innych jąder i nazwał je protonem.
Dalsze eksperymenty pokazały, że masa jądra przekracza masę protonów, spekulował, że różnica ta bierze się z nieznanych dotąd obojętnych elektrycznie cząstek, które nazwał neutronami.

\textbf{Odkrycie neutronu}.
W 1930 r. niemiecki fizyk Walther Bothe oraz jego student Herbert Becker bombardowali beryl cząstkami $\alpha$ i zauważyli, że emituje on promieniowanie, które przechodzi nawet przez 20-centymetrową ścianę z ołowiu.
Podobne eksperymenty przeprowadzali Frederic Joliot i Irena Joliot-Curie oraz James Chadwick, który w pewnej odległości od tarczy umieścił wosk parafinowy.
Energia promieniowania Roentgena wystarczała do uwolnienia elektronów z wosku, ale nie protonów.
Dlatego Chadwick stwierdził, że protony z wosku wybijane były przez inne promieniowanie obojętnych cząstek o masie zbliżonej do masy protonu, \emph{neutrony}.
W tym samym roku Dmitryj Iwanienko zasugerował, że neutrony i protony są odpowiedzialne za masę jądra.
Umożliwiło to obliczenie energii wiązań poszczególnych jąder.

\textbf{Kwantowe modele atomu}.
% TODO!


\subsection{Proton}
Proton to trwała cząstka (z grupy barionów) o ładunku +1e, masie spoczynkowej równej $\approx 1 u$ oraz promieniem około $0,833 fm \pm 0,010 fm$ (zależy, kto mierzył XD)
Zbudowany z trzech kwarków: dwóch górnych i jednego dolnego.

1.007276466621(53) Da

Protony i neutrony => nukleony?

% GREKA TODO: The word proton is Greek for "first", and this name was given to the hydrogen nucleus by Ernest Rutherford in 1920.





ELektron to cząstka o ładunku $-1e$, masie równej około $1/1836$ masy protonu
masa: % [1822.8884845(14)]-1 u
lepton

theorized ... discovered ... promien ok. 10 do -18..-22




Neutron:
2 down quarks
1 up quark
theorized, discovereed, masa, promień
\section{Elementy mechaniki kwantowej}
Pionierem fizyki kwantowej był Max Planck, który w 1900 r. przyjął, że energia fal elektromagnetycznych emitowanych przez ciało doskonale czarne jest skwantowana i proporcjonalna do częstotliwości fali.
\index[persons]{Planck, Max}%
\index[persons]{de Broglie, Louis}%
W 1905 r. Albert Einstein wyjaśnił efekt fotoelektryczny, w 1913 r. Niels Bohr wyjaśnił skwantowanie poziomów energetycznych w atomie wodoru.
\index[persons]{Einstein, Albert}%
\index[persons]{Bohr, Niels}%
Rok 1922 przyniósł odkrycie Arthura Comptona: korpuskularny charakter fotonu, zaś dwa lata później francuski fizyk Louis de Broglie przyjął, że poruszający się elektron ma własności falowe.
\index[persons]{Compton, Arthur}%
\index[persons]{de Broglie, Louis}%
W 1925 r. Werner Heisenberg, Max Born i Pascual Jordan sformułowali macierzową reprezentację mechaniki kwantowej.
\index[persons]{Heisenberg, Werner}%
\index[persons]{Born, Max}%
\index[persons]{Jordan, Pascual}%
Rok później Erwin Schrödinger opublikował konkurencyjną teorię (mechanika falowa).
Obydwa opisy okazały się równoważne.

W 1927 r. Werner Heisenberg sformułował zasadę nieoznaczoności:

\begin{theorem}[zasada nieoznaczoności Heisenberga]
	Nie można wyznaczyć jednocześnie położenia i pędu cząstki z dowolną dokładnością:
	\begin{equation}
		\sigma_x \sigma_p \ge \frac{h}{4\pi}.
	\end{equation}
\end{theorem}

orbital [łac. orbita ‘koleina’, ‘droga’]: funkcja falowa $\phi$ opisująca stan jednego elektronu, zależna od współrzędnych określających jego położenie w atomie (orbital atomowy), cząsteczce (orbital molekularny, orbital cząsteczkowy) lub krysztale.

W 1927 niemiecki astrofizyk Albrech Unsöld stwierdził, że kwadrat funkcji falowej dla podpowłoki wypełnionej całkowicie lub w połowie jest sferycznie symetryczny.

\chapter{Wiązania chemiczne}
% Jones, Atkins: 8

\chapter{Gazy}
% Jones, Atkins: 5
Molekularny charakter gazów, ciśnienie.
Prawo Boyle'a, Charlesa, Gay-Lussaca, Avogadra, stanu gazu doskonałego.
Dyfuzja i efuzja, rozkład prędkości Maxwella.

\chapter{Ciecze i ciała stałe}
% Jones, Atkins: 10
Molekularny charakter gazów, ciśnienie.
Prawo Boyle'a, Charlesa, Gay-Lussaca, Avogadra, stanu gazu doskonałego.
Dyfuzja i efuzja, rozkład prędkości Maxwella.

\chapter{Związki nieorganiczne}
% Jones, Atkins: 15
\section{Tlenki}
\subsection{Tlenek arsenu(III) \ce{As4O6}}
Arszenik, biały proszek lub kryształy, w temperaturze 200 stopni Celsjusza łatwo sublimuje, słabo reaguje z wodą, lepiej roztwarza się w kwasace, jeszcze lepiej w zasadace.
Śmiertelna trucizna, stosowany do wyrobu leków, barwników oraz trutek przeciwko owadom czy gryzoniom.

% WSIP
       %%% polish: tlenek arsenu
\subsection{Tlenek azotu(II) \ce{NO}}
Bezbarwny gaz, źle rozpuszczalny w wodzie.
Otrzymywany z pierwiastkowego tlenu i azotu w wysokiej temperaturze lub łuku elektrycznym, a w laboratorium także działając kwasem azotowym na wiórki miedzi:
$$\ce{8 HNO3 + 3 Cu -> 3 Cu(NO3)2 + 4 H2O + 2 NO}$$
Komercyjnie utlenia się amoniak w temperaturze $850\si{\celsius}$, z platyną jako katalizatorem:
$$\ce{4 NH3 + 5 O2 -> 4 NO + 6 H2O}.$$

% WSIP

% \textbf{Własności fizyczne}.
% Bezbarwny gaz.
% Gęstość $1.3402 \si{g \per cm}$, topnieje w $-163\si{\celsius}$, wrze w $-151\si{\celsius}$.

% \textbf{Własności ceemiczne}.
% Jest rodnikiem, a przez to niestabilny i bardzo reaktywny.
% W powietrzu samorzutnie reaguje z tlenem, tworząc trujący dwutlenek azotu.

% \textbf{Otrzymywanie}.

% Endotermiczna reakcja tlenu z azotem, niewymagająca katalizatora przebiega w dużo wyższej temperaturze (ponad $2000 \si{\celsius}$) nie została dotąd wykorzystana na przemysłową skalę (patrz proces Birkelanda-Eyde'a).


% \textbf{Zastosowanie}.
% Duże znaczenie biologiczne.
   %%% polish: tlenek azotu 2
\subsection{Tlenek azotu(IV) \ce{NO2}}
Czerwonobrunatny gaz, dobrze rozpuszczalny w wodzie.
Poniżej 150 stopni dimeryzuje się do bezbarwnego, ciekłego \ce{N2O4}.
Utleniacz w rakietowyce materiałace pędnyce.
   %%% polish: tlenek azotu 4
\subsection{Tlenek cynku \ce{ZnO}}
Biel cynkowa, biały proszek żółknący po ogrzanu (w wyniku powstania defektów w sieci krystalicznej), otrzymywany przez rozkład zasadowego węglanu cynku albo spalanie cynku w powietrzu.
Stosowany jako biały pigment, wypełniacz gumy do opon, jako składnik szkliw, emalii ceramicznyce oraz jako środek antyseptyczny (maść cynkowa).

% WSIP
          %%% polish: tlenek cynku
\subsection{Tlenek fosforu(V) \ce{P4O10}}
Białe kryształy, sublimujące w 360 stopniace.
Energicznie reaguje z wodą, używany jako środek suszący (gdyż jest higroskopijny).
  %%% polish: tlenek fosforu 5
\subsection{Tlenek krzemu(IV) \ce{SiO2}}
Krzemionka, podstawowy składnik piasku, wielu skał i kamieni ozdobnyce.
Najczęściej występującymi odmianami są kwarc, trydymit, krystobalit oraz opal.
Dość bierny ceemicznie, ulega tylko \ce{HF} oraz mocnym zasadom.
Zwykłe szkło zawiera około 80\% tlenku krzemu, szkło kwarcowe (prawie czysty tlenek) ma dużą odporność ceemiczną i termiczną, stosuje się je do wyrobu szkła laboratoryjnego oraz przyrządów optycznyce (przepuszcza ultrafiolet).
Żele krzemionkowe stosuje się jako środki suszące, pocełaniające, izolatory termiczne i dźwiękowe, jako dodatki do farb i lakierów.

% WSIP
       %%% polish: tlenek krzemu
\subsection{Tlenek magnezu \ce{MgO}}
Magnezja palona, otrzymany w niższej temperaturze: biały proszek łatwo reagujący z wodą oraz kwasami, pocełania tlenek węgla(II) i wilgoć z powietrza.
Otrzymany w wyższej: tworzy duże kryształy, odporne na działanie wody i kwasów, znane jako minerał peryklaz.
Używany do wyrobu materiałów ogniotrwałyce, cementu magnezjowego, pigmentów używanyce do produkcji emalii.

% https://pl.wikipedia.org/wiki/Magnezja

% WSIP
     %%% polish: tlenek magnezu
\subsection{Tlenek manganu(IV) \ce{MnO2}}
Braunsztyn, dodawany w przemyśle szklarskim do stopionej masy szklanej, by odbarwić szkło.
Stosowany także w ogniwace (bateriace jednorazowyce), np. w ogniwace Leclanceego albo alkalicznyce.
  %%% polish: tlenek manganu 4
\subsection{Tlenek ołowiu(II) \ce{PbO}}
Glejta, nierozpuszczalne w wodzie ciało krystaliczne, powyżej 400 stopni Celsjusza trwała jest odmiana żółta (massicot), w niższyce czerwona (litargit).
Używany do wyrobu szkieł ołowiowyce.

% WSIP
       %%% polish: tlenek ołowiu 2
\subsection{Tlenek diołowiu(II) ołowiu(IV) \ce{Pb3O4}}
Minia ołowiana, czerwone ciało stałe stosowane do wyrobu farb antykorozyjnyce oraz kitów uszczelniającyce.

% WSIP
    %%% polish: tlenek ołowiu 2, 4
\subsection{Tlenek azotu(I) \ce{N2O}}
Bezbarwny gaz otrzymywany przez rozkład azotanu(V) amonu. 
Rozkłada się z wydzieleniem tlenu (podtrzymując palenie).
Dawniej używany jako gaz rozweselający.

% WSIP
    %%% polish: tlenek siarki 1
\subsection{Tlenek siarki(IV) \ce{SO2}}
Gaz o ostrym zapaceu, trujący w stężeniu wyższym niż$5 \cdot 10^{-4} \%$.
Otrzymywany przez spalanie siarki albo prażenie rud siarczkowyce lub anhydrytu.
Służy do bielenia, konserwacji artykułów spożywczyce, dezynfekcji.
     %%% polish: tlenek siarki 4
\subsection{Tlenek siarki(VI) \ce{SO3}}
Bezbarwyn gaz o silnyce właściwościace odwadniającyce, utleniacz.
Atakuje drogi oddeceowe oraz błony śluzowe.
     %%% polish: tlenek siarki 6
\subsection{Tlenek wanadu(V) \ce{V2O5}}
\textbf{Własności fizyczne}. 
Żółtoczerwony proszek, gęstość $3.4 \si{g \per cm}$, topnieje w $690\si{\celsius}$.
Słabo rozpuszcza się w wodzie.
Występuje jako niezwykle rzadki minerał, szczerbinait, prawie zawsze znajduje się w wyziewace wulkanicznyce.
Trihydrat znany jest także jako navajoit.

\textbf{Własności ceemiczne}.
Reaguje z mocnymi nieredukującymi kwasami, tworząc roztwory zawierające blade żółte sole:
$$\ce{V2O5 + 2 HNO3 -> 2 VO2(NO3) + H2O}$$

Reaguje także z mocnymi zasadami:
$$\ce{V2O5 + 6 NaOH -> 2 Na3VO4 + 3 H2O}.$$
Jeśli użyje się nadmiaru wodnego roztworu wodorotlenku sodu, produkt jest bezbarwną solą, \ce{Na3VO4}.
Powolne dodawanie kwasu zmienia kolor przez pomarańczowy do czerwonego, zanim brązowy uwodniony \ce{V2O5} wytrąca się w formie osadu okolo pH 2.
Taki roztwór zawiera głównie jony \ce{HVO4^{2}-} and \ce{V2O7^{4}-} między pH 9 oraz 13, ale poniżej pH 9, dominują ,,egzotyczne'' jony \ce{V4O12^{4}-}, \ce{HV10O28^{5}-}.

Trujący, mutagenny.
Mieszanina tlenku wanadu(V) z tlenkiem wanadu(III) podgrzana daje tlenek wanadu(IV), niebieskie ciało stałe.

\textbf{Otrzymywanie}.
Przemysłowo produkowany jako czarny pył.
Ruda wanadu albo bogaty w wanad osad potraktowany węglanem sodu oraz solami amonowymi daje metawanadan sodu, \ce{NaVO3}.
Następnie zakwasza się ten półprodukt do pH 2-3 przy użyciu kwasu siarkowego, otrzymując czerwony osad.
Stopiony w $690\si{\celsius}$ daje surowy tlenek wanadu.

Tlenek wanadu można otrzymać także ogrzewając metaliczny wanad w nadmiarze tlenu, ale produkt jest zanieczyszczony innymi, niższymi tlenkami.
Lepszą metodą jest rozkład termiczny metawanadanu amonu w temperaturze $500\si{\celsius}$ do $550\si{\celsius}$:
$$\ce{2 NH4VO3 -> V2O5 + 2 NH3 + H2O}.$$

\textbf{Zastosowanie}.
Przede wszystkim (pod względem ilości) wykorzystywany do otrzymania żelazowanadu, dodatku w produkcji stali.
Katalizator przy produkcji kwasu siarkowego albo utleniania dwutlenku siarki do tritlenku.
Stosowany także przy produkcji szkła absorbującego promieniowanie nadfioletowe.
% Vanadium(V)-oxid wird auce verwendet, um Weißglas undurcelässiger für UV-Licet zu maceen. Dazu wird es der Glasscemelze zugesetzt, das fertige Glas ist weder von außen noce von innen besceicetet. Flasceen aus diesem Glas werden vor allem für Bier verwendet, um den Licetgescemack zu vermeiden.
Z przemysłowego punktu widzenia, to najważniejszy związek wanadu, główny prekursor różnyce stopów wanadu.
    %%% polish: tlenek wanadu 5
\subsection{Wodorotlenek wapnia \ce{Ca(OH)2}}
% Unprotected exposure to Ca(OH)2 can cause severe skin irritation, chemical burns, blindness or lung damage or rashes [5]
% Limewater is the common name for a saturated solution of calcium hydroxide.
% One significant application of calcium hydroxide is as a flocculant, in water and sewage treatment. It forms a fluffy charged solid that aids in the removal of smaller particles from water, resulting in a clearer product.
% Another large application is in the paper industry, where it is an intermediate in the reaction in the production of sodium hydroxide. This conversion is part of the causticizing step in the Kraft process for making pulp.[8] In the causticizing operation, burned lime is added to green liquor, which is a solution primarily of sodium carbonate and sodium sulfate produced by dissolving smelt, which is the molten form of these chemicals from the recovery furnace.
% Unprotected exposure to Ca(OH)2 can cause severe skin irritation, chemical burns, blindness or lung damage or rashes [5]
\textbf{Informacje ogólne}
Jako minerał, portlandyt, występuje stosunkowo rzadko: w skałach wulkanicznych oraz przeobrażonych.

\textbf{Własności fizyczne}.
Bezwonne, bezbarwne kryształy lub biały proszek.
Rozkłada się podczas topnienia w $580 \si{\celsius}$ na tlenek wapnia i wodę, gęstość $2.211 \si{g \per cm}$.
Słabo rozpuszczalny w wodzie.

\textbf{Własności chemiczne}.
Jego wodny roztwór nazywa się wodą wapienną, jest dość mocną zasadą (pH ok. 12.4) o żrącym działaniu, która w obecności dwutlenku węgla mętnieje wskutek wytrącania się węglanu wapnia.
(Roztwór) atakuje niektóre metale (glin), chroniąc jednocześnie inne przed korozją (żelazo, stal) przez pasywację.
Stosowany przy produkcji amoniaku: $$\ce{Ca(OH)2 + 2NH4Cl ->  2NH3 + CaCl2 + 2H2O}$$

\textbf{Otrzymywanie}.
W laboratorium można wymieszać wodne roztwory chlorku wapnia oraz wodorotlenku sodu.
Nazwa wapno gaszone (albo lasowane) pochodzi od przemysłowej metody otrzymywania: $$\ce{CaO + H2O -> Ca(OH)2}$$

\textbf{Zastosowanie}.
W budownictwie jako spoiwo zapraw murarskiej i tynkarskiej, jako baza i lepiszcze farb malarskich o właściwościach dezynfekujących.
W cukrownictwie do oczyszczania soku buraczanego, jako zmiękczacz wody.
W energetyce do odsiarczania spalin.
Składnik cementu stomatologicznego oraz konserwant żywności (E526).

       %%% polish: tlenek wapnia
\subsection{Tlenek węgla(II) \ce{CO}}
Bezbarwny, bezwonny i lżejszy od powietrza gaz (czad).
Przyłącza się do żelaza w hemoglobinie, a przez to trujący (blokuje transport tlenu).
Używany w metalurgii w wysokice temperaturace jako silny reduktor do produkcji metali z tlenków.

% WSIP
     %%% polish: tlenek węgla 2
\subsection{Tlenek węgla(IV) \ce{CO2}}
Bezbarwny, bezwonny, niepalny i niepodtrzymujący palenia, nietrujący gaz.
Gęstszy od powietrza, dobrze rozpuszcalny w wodzie,
W laboratorium otrzymywany w aparacie Kippa (kwas solny działający na węglan wapnia), spalanie koksu w nadmiarze powietrza lub podczas fermentacji.
Ciekły i stały są stosowane jako substancja cełodząca, gazowy do produkcji mocznika, napojów gazowanyce, w gaśnicace.

% WSIP
     %%% polish: tlenek węgla 4
\subsection{Tlenek żelaza(III) \ce{Fe2O3}}
Związek o dwóce odmianace, trwalsza ($\alpha$) tworzy czerwonobrunatne kryształy i występuje jako minerał hematyt, używana do wyrobu farb oraz polerowania w jubilerstwie.
Mniej trwała ($\gamma$) tworzy się podczas prażenia tlenku \ce{Fe3O4} w próżni, ma własności ferromagnetyczne i używa się jej do produkcji taśm magnetycznyce.
% WSIP
      %%% polish: tlenek żelaza 3

\section{Kwasy}
% kwasy beztlenowe
\subsection{Kwas solny \ce{HCl}}
\textbf{Własności fizyczne}.
Czysty kwas jest bezbarwny, techniczny ma żółtawe zabarwienie, gdyż jest zanieczyszczony jonami żelaza.
Maksymalne stężenie wynosi 45\%, ze stężonego kwasu ulatnia się gazowy chlorowodór, który reagując z wilgocią z powietrza tworzy mgłę; stąd kwas określa się jako dymiący.
Poniżej 30\% już nie jest dymiący, ale wciąż lotny.
Ma wtedy gęstość $1.149 \si{g \per cm^3}$, topnieje w $-52\si{\celsius}$, wrze w $90\si{\celsius}$.
Wyraźny, gryzący zapach.

\textbf{Własności chemiczne}.
Jeden z najmocniejszych kwasów nieorganicznych.
Nie ma właściwości utleniających.

\textbf{Otrzymywanie}.
W XV wieku Basilius Valentinus otrzymał go z soli kamiennej oraz siarczanu żelaza(II).

% TODO: During the Industrial Revolution in Europe, demand for alkaline substances increased. A new industrial process developed by Nicolas Leblanc of Issoudun, France enabled cheap large-scale production of sodium carbonate (soda ash). In this Leblanc process, common salt is converted to soda ash, using sulfuric acid, limestone, and coal, releasing hydrogen chloride as a by-product. Until the British Alkali Act 1863 and similar legislation in other countries, the excess HCl was vented into the air. After the passage of the act, soda ash producers were obliged to absorb the waste gas in water, producing hydrochloric acid on an industrial scale.[13][25]

% TODO: In the 20th century, the Leblanc process was effectively replaced by the Solvay process without a hydrochloric acid by-product. Since hydrochloric acid was already fully settled as an important chemical in numerous applications, the commercial interest initiated other production methods, some of which are still used today. After the year 2000, hydrochloric acid is mostly made by absorbing by-product hydrogen chloride from industrial organic compounds production.[13][25][26]

\textbf{Zastosowanie}.
Jest jednym z najważniejszych kwasów w przemyśle (m.in. przemysł włókienniczy, tworzyw sztucznych, farmaceutyczny, garbarstwo, cukrownictwo, produkcja żelatyny, barwników, ekstrakcja rud).
Trawienie stali: usuwa rdzę lub tlenki żelaza przed innymi procesami, takimi jak tłoczenie, walcowanie, cynkowanie. Używa się do tego technicznego kwasu o stężeniu 18\%.
Produkcja związków organicznych (chlorek winylu, dichlorek etylenu) dla PVC, bisfenolu A, poliwęglanów, kwasu askorbinowego i innych.
Produkcja związków nieorganicznych: chlorku wapnia (zapobiega oblodzeniom na drogach), chlorku niklu (galwanostegia), chlorku cynku (cynkowanie, produkcja baterii).
Prekursor narkotykowy 3 kategorii.
 % Zwyczajowa nazwa „kwas solny” pochodzi od dawnej metody jego otrzymywania ze stężonego kwasu siarkowego i soli kamiennej (kwas z soli).

% kwasy tlenowe
\subsection{Kwas azotowy(V) \ce{HNO3}}
\textbf{Własności fizyczne}.
Dymiąca bezbarwna ciecz, topnieje w $-41\si{\celsius}$, wrze w $86\si{\celsius}$.
Pod wpływem światła lub ogrzewania rozkłada się na tlenki o żółtej barwie.
Tworzy z wodą azeotrop.

\textbf{Własności chemiczne}.
Jego pary są toksyczne.
Czysty kwas jest wybuchowy.
Reaguje z większością metali, na jego działanie odporne są złoto, platyna, rod, iryd, tantal oraz niektóre metale nieszlachetne ulegające pasywacji (żelazo, glin, itd.).
W reakcji stężonego kwasu z białkami powstają żółto zabarwione produkty (reakcja ksantoproteinowa, stosowana w analizie do wykrywania białek).
Stężony działa żrąco na tkaniny i skórę.

\textbf{Otrzymywanie}.
Metoda katalityczna: utlenianie amoniaku do tlenku azotu(II), który jest dalej utleniany  do tlenku azotu(IV) tlenem z powietrza, a następnie absorbowany w wodzie.

\textbf{Zastosowanie}.
Produkcja nawozów, włókien sztucznych (nylony), materiałów wybuchowych (nitrogliceryna), barwników, lekarstw, azotanu(V) srebra(I) dla przemysłu fotograficznego, jako  utleniacz w rakietowych materiałach pędnych.
      %%% polish: kwas azotowy ?
\subsection{Kwas azotowy(III) \ce{HNO2}}
W niewielkich ilościach obecny w wodzie deszczowej oraz ściekach.

\textbf{Własności chemiczne}.
W bardziej stężonych roztworach wodnych ulega samorzutnemu rozpadowi na kwas azotowy(V) oraz tlenek azotu(II).
Kwas i jego sole mogą działać jak utleniacze i reduktory.

     %%% polish: kwas azotowy ?
\subsection{Kwas ortofosforowy \ce{H_3PO_4}}
\textbf{Własności fizyczne}.
Bezbarwne ciało stałe (kryształy), gęstość $2.03 \si{g \per cm}$, topnieje w $42\si{\celsius}$, wrze w $407\si{\celsius}$.
Najczęściej występuje jako lepki, nielotny, bezbarwny i bezwonny roztwór wodny 85\%.
Higroskopijny.
% but still pourable!
Rozpuszczalny także w etanolu.

\textbf{Własności chemiczne}.
Słaby kwas trójprotonowy.
Roztwór wodny pomimo to drażni skórę oraz uszkadza oczy.
Gwałtownie reaguje z zasadami, polimeryzuje się pod wpływem azozwiązków i epoksydów.
Rozkłada się podczas kontaktu z alkoholami, aldehydami, cyjankami, ketonami, fenolami, estrami, siarczkami, wydzielając przy tym toksyczne opary.

\textbf{Otrzymywanie}.
W przemyśle stosuje się dwie metody.
Proces mokry polega na potraktowaniu kwasem siarkowym minerałów zawierających fosforan, np. hydroksyapatytu: $$\ce{Ca5(PO4)3OH + 5 H2SO4 -> 3 H3PO4 + 5 CaSO4 v + H2O}$$ albo apatytu, fosforytu, czasami także zmielonych kości zwierząt:
$$\ce{Ca3(PO4)2 + 3 H2SO4 -> 2 H3PO4 + 3 CaSO4}$$
Wydzielający się w tej reakcji siarczan wapnia można łatwo oddzielić od kwasu fosforowego przez filtrowanie, gdyż nie rozpuszcza się on w wodzie.
Alternatywnym źródłem jest fluoroapatyt, gdzie powstaje nierozpuszczalny fluorek \ce{Na2SiF6}.

\textbf{Zastosowanie}. Głównie (90\%) do produkcji nawozów sztucznych takich jak superfosfat podwójny. W przemyśle spożywczym jako regulator kwasowości E338.
% Stosowany jest też do wytwarzania fosforanowych powłok ochronnych na metalach, do wytwarzania wielu środków farmaceutycznych, oczyszczania soków w cukrownictwie, odkamieniania armatury w ciepłownictwie, jako płyn do lutowania, w stomatologii, do wyrobu kitów porcelanowych, w lecznictwie i laboratoriach analitycznych. Jest także składnikiem fosolu – odrdzewiacza do stali.
  %%% polish: kwas fosforowy ?
\subsection{Kwas siarkowy(IV) \ce{H2SO3}}
Kwas o średniej mocy występujący tylko w wodnym roztworze, powstaje podczas rozpuszczania tlenku siarki(IV) w wodzie.
Ma właściwości redukujące.
Używany do produkcji papieru, w browarnictwie, metalurgii, jako środek konserwujący oraz dezynfekujący.
   %%% polish: kwas siarkowy ?
\subsection{Kwas siarkowy(VI) \ce{H2SO4}}
\textbf{Własności fizyczne}.
Bezbarwna, oleista ciecz, topnieje w $10\si{\celsius}$, wrze w $320\si{\celsius}$ (z częściowym rozkładem).
Miesza się z wodą w każdym stosunku, podczas czego może wydzielić się tak dużo ciepła, że temperatura mieszaniny podniesie się o ponad $100\si{\celsius}$.

\textbf{Własności chemiczne}.
Gorący, stężony kwas jest silnym utleniaczem, reaguje z miedzią, srebrem, rtęcią, utlenia też niemetale (redukując się przy tym do tlenku siarki(IV)).
Rozcieńczony nie ma właściwości utleniających.
Stężony jest silnym środkiem odwadniającym, zwęgla cukier i drewno, powoduje oparzenia i trudno gojące się rany.

\textbf{Otrzymywanie}.
Metoda kontaktowa: katalityczne utlenienie tlenku siarki(IV).

\textbf{Zastosowanie}.
Produkcja innych kwasów, nawozów sztucznych (superfosfat), osuszanie gazów, w garbarstwie i przemyśle spożywczym, włókienniczym i papierniczym, jako elektrolit w akumulatorach ołowiowych.
    %%% polish: kwas siarkowy ?
\subsection{Kwas węglowy \ce{H2CO3}}
Słaby kwas powstający przez rozpuszczenie dwutlenku węgla w wodzie, zawierają go wszystkie napoje gazowane.
Tylko dwa promile cząsteczek dwutlenku reagują z wodą.
Przez lata wierzono, że nie istnieje jako czysty związek, aż w 1991 naukowcy z NASA otrzymali próbkę w stanie stałym.
    %%% polish: kwas węglowy


\section{Wodorotlenki}
%\input{template}                      %%% polish: wodorotlenek cynku \ch{Zn(OH)2} wodorotlenek cynku
\subsection{Wodorotlenek glinu \ce{Al(OH)3}}
\textbf{Własności fizyczne}.
Biały bezwonny, amorficzny proszek o gęstości $2.42 \si{g \per cm^3}$.
Temperatura topnienia $300 \si{\celsius}$.
Nie rozpuszcza się w etanolu ani w wodzie.
Naturalnie występuje jako minerał w czterech odmianach polimorficznych: gibbsyt, bajeryt, doyleite, nordstrandyt.

\textbf{Własności chemiczne}.
Amfoteryczny (kwasowo-zasadowy), pH powyżej 7.
Niepalny.

\textbf{Otrzymywanie}.
Przemysłowo niemal w całości otrzymuje się go w procesie Bayera (1887 r.), gdzie boksyt rozpuszcza się w roztworze wodorotlenku sodu do około $270 \si{\celsius}$.
Odpady (czerwony szlam) usuwa się, a następnie strąca wodorotlenek z pozostałego roztworu glinianu sodu.

\textbf{Zastosowanie}.
Środek opóźniający palenie: rozpada się w około $180 \si{\celsius}$, pochłaniając część ciepła i oddając parę wodną.
Środek tłumiący dym, głównie w poliestrach, akrylach, EVA, PVC, żywicach epoksydowych. 
W farmacji: środek na nadkwaśność żołądka (zobojętnia kwas, nierozpuszczalny, nie zwiększa pH powyżej 7), substancja ścierająca i polerująca przy produkcji past do zębów.
Produkcja papieru, mydła, kosmetyków.
  %%% polish: wodorotlenek glinu
%\input{template}                      %%% polish: wodorotlenek potasu \ch{KOH} wodorotlenek potasu (potaż żrący)
%\input{template}                      %%% polish: wodorotlenek sodu \ch{NaOH} wodorotlenek sodu (soda żrąca)
\subsection{Wodorotlenek wapnia \ce{Ca(OH)2}}
% Unprotected exposure to Ca(OH)2 can cause severe skin irritation, chemical burns, blindness or lung damage or rashes [5]
% Limewater is the common name for a saturated solution of calcium hydroxide.
% One significant application of calcium hydroxide is as a flocculant, in water and sewage treatment. It forms a fluffy charged solid that aids in the removal of smaller particles from water, resulting in a clearer product.
% Another large application is in the paper industry, where it is an intermediate in the reaction in the production of sodium hydroxide. This conversion is part of the causticizing step in the Kraft process for making pulp.[8] In the causticizing operation, burned lime is added to green liquor, which is a solution primarily of sodium carbonate and sodium sulfate produced by dissolving smelt, which is the molten form of these chemicals from the recovery furnace.
% Unprotected exposure to Ca(OH)2 can cause severe skin irritation, chemical burns, blindness or lung damage or rashes [5]
\textbf{Informacje ogólne}
Jako minerał, portlandyt, występuje stosunkowo rzadko: w skałach wulkanicznych oraz przeobrażonych.

\textbf{Własności fizyczne}.
Bezwonne, bezbarwne kryształy lub biały proszek.
Rozkłada się podczas topnienia w $580 \si{\celsius}$ na tlenek wapnia i wodę, gęstość $2.211 \si{g \per cm}$.
Słabo rozpuszczalny w wodzie.

\textbf{Własności chemiczne}.
Jego wodny roztwór nazywa się wodą wapienną, jest dość mocną zasadą (pH ok. 12.4) o żrącym działaniu, która w obecności dwutlenku węgla mętnieje wskutek wytrącania się węglanu wapnia.
(Roztwór) atakuje niektóre metale (glin), chroniąc jednocześnie inne przed korozją (żelazo, stal) przez pasywację.
Stosowany przy produkcji amoniaku: $$\ce{Ca(OH)2 + 2NH4Cl ->  2NH3 + CaCl2 + 2H2O}$$

\textbf{Otrzymywanie}.
W laboratorium można wymieszać wodne roztwory chlorku wapnia oraz wodorotlenku sodu.
Nazwa wapno gaszone (albo lasowane) pochodzi od przemysłowej metody otrzymywania: $$\ce{CaO + H2O -> Ca(OH)2}$$

\textbf{Zastosowanie}.
W budownictwie jako spoiwo zapraw murarskiej i tynkarskiej, jako baza i lepiszcze farb malarskich o właściwościach dezynfekujących.
W cukrownictwie do oczyszczania soku buraczanego, jako zmiękczacz wody.
W energetyce do odsiarczania spalin.
Składnik cementu stomatologicznego oraz konserwant żywności (E526).

    %%% polish: wodorotlenek wapnia \ch{Ca(OH)2} wodorotlenek wapnia (wapno gaszone)
\section{Sole}
Sole to związki chemiczne zbudowane z kationów metali lub kationu amonu \ce{NH4+} oraz anionów reszt kwasowych.
Wyróżnia się trzy główne rodzaje soli:
\begin{compactitem}
\item \important{sole obojętne}: \important{sole proste} (zawierające jeden rodzaj kationów i jeden rodzaj anionów), na przykład \ce{CuBr2}; \important{sole podwójne} zawierające dodatkowo jeeden rodzaj kationów lub anionów, na przykład \ce{AlNa(SO4)2} oraz \important{hydraty} czyli sole uwodnione, zawierające cząsteczki wody wbudowane w sieć krystaliczną, na przykład \ce{CuSO4 . 5H20};
\item \important{wodorosole} zawierają aniony powstające podczas stopniowej dysocjacji kwasów wieloprotonowych, na przykład \ce{NH4HSO4}
\item \important{hydroksosole} zawierają aniony wodorotlenkowe, aniony reszt kwasowych i kationy metali (lub amonu), na przykład \ce{CaCl(OH)} to chlorek wodorotlenek wapnia.
\end{compactitem}

Sposoby otrzymywania:
\begin{compactitem}
\item reakcja zobojętniania: \ce{KOH + HCl -> KCl + H20},
\item reakcja tlenku metalu z kwasem: \ce{ZnO + 2HNO3 -> Zn(NO3)2 + H2O},
\item reakcja tlenku kwasowego z zasadą: \ce{SO2 + 2KOH -> K2SO3 + H2O},    
\item reakcja tlenku zasadowego z tlenkiem kwasowym: \ce{BaO + CO2 -> BaCO3},
\item reakcja metalu z niemetalem: \ce{Cu + Cl2 -> CuCl2},
\item reakcja metalu aktywniejszego od wodoru z kwasem: \ce{Fe + H2SO4 -> FeSO4 + H2 ^},
\item reakcja dwóch wodnych roztworów soli, z których jedna zawiera kation, a druga anion soli trudno rozpuszczalnej: \ce{AgNO3 + KCl -> AgCl v + KNO3}
\end{compactitem}

Dysocjacja soli prostych przebiega jednostopniowo, wodorosole dysocjują wielostopniowo.

\chapter{Stechiometria}
Molekularny charakter gazów, ciśnienie.
Prawo Boyle'a, Charlesa, Gay-Lussaca, Avogadra, stanu gazu doskonałego.
Dyfuzja i efuzja, rozkład prędkości Maxwella.
% Jones, Atkins: 2, 4

\chapter{Reakcje chemiczne}

\chapter{Roztwory}

\chapter{Kinetyka chemiczna}

\chapter{Równowaga chemiczna}

\chapter{Termochemia}

\chapter{Elektrochemia}

\chapter{Reakcje w wodnych roztworach elektrolitów}

\chapter{Charakterystyka pierwiastków chemicznych}

\chapter{Chemia jądrowa}

\chapter{Chemia organiczna}


\raggedright
\bibliographystyle{alpha}
\bibliography{chemistry_notes}

\indexprologue{\small Tekst prologu I.}
\printindex

\indexprologue{\small Tekst prologu II.}
\printindex[persons]

\end{document}