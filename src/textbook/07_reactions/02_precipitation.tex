\section{Reakcje strącania} 
Osad –- stała, dająca się łatwo oddzielić mechanicznie część ciekłej mieszaniny niejednorodnej, nietworząca trwałego układu koloidalnego. Proces celowego generowania osadu z roztworu nazywa się strącaniem lub wytrącaniem. % wiki
Reakcja strącania jest reakcją powstawania nierozpuszczalnego produktu stałego w wyniku zmieszania dwóch roztworów elektrolitów.
Na przykład: \ce{AgNO3 + NaCl -> AgCl v + NaNO3} (reakcja podwójnej wymiany).
Pełne równanie jonowe:
\ce{Ag+ + NO3- + Na+ + Cl- -> AgCl v  + Na+ + NO3-} uwzględnia wszystkie jony znajdujące się w roztworze.
Tutaj jony towarzyszące (sodu, azotowo???) nie biorą udziału w reakcji, zatem możemy zapisać również skrócone równanie jonowe: \ce{Ag + Cl -> AgCl v}.

Reakcje strącania to jedna z metod otrzymywania nowych związków: dobiera się roztwory wyjściowe tak, by po ich zmiesznaiu powstał pożądany osad. Następnie można go oddzielić od roztworu przez sączenie.