\section{Wprowadzenie}
\label{section_chemical_reactions_intro}%
Substancje wyjściowe w reakcji chemicznej noszą nazwę \important{substratów}, substancje powstające w jej wyniku to \important{produkty}.
Substraty i produkty obejmuje się jako \important{reagenty}.
\important{Odczynniki} to substancje stosowane w laboratorium chemicznym (?).

Wyrażenie \ce{Na + H2O -> NaOH + H2} nazywamy \important{równaniem szkieletowym}, określa ono substancje biorące udział w reakcji, ale nie ich ilości.
Zbilansowane równanie chemiczne to takie, w którym są też \important{współczynniki stechiometryczne}: \ce{2Na + 2H2O -> 2NaOH + H2}.
Czasami podaje się też stan skupienia: (s) stały, (c) ciekły, (g) gazowy i (aq) roztwór wodny.
Aby podkreślić, że reakcja wymaga wysokiej temperatury, nad kreską pisze się grecką literę Delta:
\begin{center}
    \ce{CaCO3(s) ->[\Delta] CaO + CO2 ^}. 
\end{center}

Strzałka skierowana do góry \ce{ ^} oznacza ulatnianie się gazowego produktu reakcji do atmosfery.
Strzałka skierowana w dół \ce{v} oznacza strącanie się nierozpuszczalnego, stałego produktu (osadu) z roztworu, w którym zachodziła reakcja.

TODO: substraty, produkty, współczynniki stechiometryczne
Synteza, analiza, wymiana
