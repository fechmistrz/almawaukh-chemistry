\section{Reakcje utleniania-redukcji}
\label{section_chemical_reactions_redox}
Aby móc powiedzieć, czym są reakcje \important{redoks} (utleniania-redukcji), trzeba podać pewną definicję.
\important{Stopień utlenienia} to umowne pojęcie określające liczbę ładunków elementarnych, które można by przypisać atomowi pierwiastka w określonym związku, gdyby cząsteczki tego związku miały budowę jonową.

Komentarz autora: podobnie jak Tomasz Pluciński\footnote{\url{http://www.tomek.strony.ug.edu.pl/rowna.htm}}, nie rozumiem, czemu większość chemików nie wyobraża sobie chemii bez stopni utlenienia.
\emph{,,W wielu innych cząsteczkach stopień utlenienia atomu nie ma nic wspólnego ze stanem wiązalności tego atomu. (...) Obliczanie stopni utlenienia atomów węgla w związkach organicznych prowadzi do karkołomnych rezultatów. (...) Aniele Stróżu mój, chroń mnie przed diabelską pokusą przypisywania stopniom utlenienia jakiegokolwiek realnego sensu chemicznego oraz przed kontaktem z egzaminatorem, który tego oczywistego faktu nie przyjmuje do wiadomości.''}

Stopień utlenienia pierwiastka w stanie wolnym wynosi 0, suma stopni atomów w cząsteczce obojętnej wynosi 0, a jonu równa się ładunkowi tego jonu.
Stopień utlenienia fluoru wynosi -I, tlenu -II (poza nadtlenkami, podtlenkami albo difluorkiem tlenu), wodoru I (poza wodorkami litowców i berylowców, gdzie wodór ma stopień -I).
Stopnie utlenienia oznacza się cyframi rzymskimi, chyba że wynosi zero lub jest ułamkiem :/.

Kiedyś utlenianiem nazywano reakcję z tlenem, dzisiaj chemicy definiują utlenianie jako utratę elektronów, nawet gdy w reakcji nie uczestniczy tlen.
Podobnie kiedyś redukcją określano otrzymywanie metalu z jego tlenku, np. \ce{Fe2O3 + 3H2 -> 2Fe + 3H2O}, obecnie się tego nie robi.

\important{Reakcja redoks} to taka, w której dochodzi do redukcji (atom lub ich grupa przechodzi z wyższego stopnia utlenienia na niższy) oraz utleniania (z niższego na wyższy), zwanych reakcjami połówkowymi.
Należy pamiętać, że reakcje połówkowe nie są rzeczywistym mechanizmem danej reakcji!
Reakcje te mogą zachodzić powoli (powstawanie rdzy) albo szybko (spalanie paliwa).

Często używanymi \important{utleniaczami} są tlen, nadtlenek wodoru, roztwór siarczanu żelaza (II) w nadtlenku wodoru (odczynnik Fentona), kwas azotowy (V), siarkowy (VI), nadsiarkowy, azotany, nadmanganiany, dichromiany.
Silnymi \important{reduktorami} są m.in.: lit, sód, magnez, wodór, żelazo, cynk, tlenek węgla, borowodorek sodu \ce{NaBH4}, tiosiarczany.

% SŁOWNICZEK:  oxidizing agents, oxidants, or oxidizers.  Oxygen is the quintessential oxidizer.