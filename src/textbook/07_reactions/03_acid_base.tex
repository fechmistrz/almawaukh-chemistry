\section{Reakcje kwasów i zasad}
W stanie czystym wiele kwasów to związki cząsteczkowe, ale w roztworze wodnym tworzą jony.
Np. \ce{HCl + H2O -> H3O+ + Cl-}.

Kwas Arrheniusa to związek, który zaweiera wodów i uwalnia w wodzie jony wodorowe.

Zasada Arrheniusa to związek wytwarzający w wodzie jony wodorotlenkowe.

Na przykład NaOH ale też NH3, bo \ce{NH3 + H2O -> NH4+ + OH-}.

Kwasy mocne/słabe.
Większość kwasów jest słaba, mocne są solny, azotowy, siarkowy
HCl, HBr, HI, HClO3/O4

Podobnie: zasady mocne/słabe
Pospolite mocne zasady to tlenki i wodorotlenki litowców i wapniowców, należą do nich NaOH, CaO, BaOH2

Inna nazwa: reakcje zobojętniania
kwa s+ z asasad a -> sól + woda

Przeniesienie protonu

Tlenki zasadowe/kwasowe