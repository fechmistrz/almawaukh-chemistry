\section{Sole}
Sole to związki chemiczne zbudowane z kationów metali lub kationu amonu \ce{NH4+} oraz anionów reszt kwasowych.
Wyróżnia się trzy główne rodzaje soli:
\begin{compactitem}
\item \important{sole obojętne}: \important{sole proste} (zawierające jeden rodzaj kationów i jeden rodzaj anionów), na przykład \ce{CuBr2}; \important{sole podwójne} zawierające dodatkowo jeeden rodzaj kationów lub anionów, na przykład \ce{AlNa(SO4)2} oraz \important{hydraty} czyli sole uwodnione, zawierające cząsteczki wody wbudowane w sieć krystaliczną, na przykład \ce{CuSO4 . 5H20};
\item \important{wodorosole} zawierają aniony powstające podczas stopniowej dysocjacji kwasów wieloprotonowych, na przykład \ce{NH4HSO4}
\item \important{hydroksosole} zawierają aniony wodorotlenkowe, aniony reszt kwasowych i kationy metali (lub amonu), na przykład \ce{CaCl(OH)} to chlorek wodorotlenek wapnia.
\end{compactitem}

Sposoby otrzymywania:
\begin{compactitem}
\item reakcja zobojętniania: \ce{KOH + HCl -> KCl + H20},
\item reakcja tlenku metalu z kwasem: \ce{ZnO + 2HNO3 -> Zn(NO3)2 + H2O},
\item reakcja tlenku kwasowego z zasadą: \ce{SO2 + 2KOH -> K2SO3 + H2O},    
\item reakcja tlenku zasadowego z tlenkiem kwasowym: \ce{BaO + CO2 -> BaCO3},
\item reakcja metalu z niemetalem: \ce{Cu + Cl2 -> CuCl2},
\item reakcja metalu aktywniejszego od wodoru z kwasem: \ce{Fe + H2SO4 -> FeSO4 + H2 ^},
\item reakcja dwóch wodnych roztworów soli, z których jedna zawiera kation, a druga anion soli trudno rozpuszczalnej: \ce{AgNO3 + KCl -> AgCl v + KNO3}
\end{compactitem}

Dysocjacja soli prostych przebiega jednostopniowo, wodorosole dysocjują wielostopniowo.