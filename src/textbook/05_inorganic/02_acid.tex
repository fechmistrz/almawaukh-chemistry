\section{Kwasy}
1923 -- Bronsted (donor protonu), to samo Lowry.
Obejmują kwasy Arrheniusa.
Deprotonowanie.
Mocne kwasy?



Kwasy to związki chemiczne zbudowane z wodoru oraz reszty kwasowej, niemetalu (kwasy beztlenowe) lub grupy atomów zawierającej tlen oraz przynajmniej jeden inny pierwiastek chemiczny (kwasy tlenowe).

\important{Kwasy beztlenowe} to wodne roztwory wodorków niemetali albo cyjanowodoru: \ce{HCN}.
Aby je otrzymać, rozpuszcza się w wodzie gazowe wodorki albo reaguje kwasy z niektórymi solami, np. \ce{H2SO4 + 2 NaF -> 2 HF + Na2SO4}.

Sposoby otrzymywania \important{kwasów tlenowych}:
\begin{compactitem}
\item reakcja tlenków niemetali z wodą: \ce{N2O5 + H20 -> 2HNO3},
\item reakcja kwasów z solami: \ce{Na2SiO3 + H2SO4 -> H2SiO3 v + Na2SO4}.
\end{compactitem}

W roztworach wodynch kwasy w różnym stopniu ulegają \important{dysocjacji elektrolitycznej}, więc te roztwory wykazują zdolność do przewodzenia prądu elektrycznego, są elektrolitami.

kwasy: beztlenowe, tlenowe

kwasy wodne to elektrolity, bo ulegają dysocjacji elektrolitycznej = jonowej
mocne - slabe


metale szlacheten są najmniej aktywne nie wypierają wodoru z kkwasów, tylko z kwasaami utleniającymi
