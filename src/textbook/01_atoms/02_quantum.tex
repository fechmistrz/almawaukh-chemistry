\section{Elementy mechaniki kwantowej}
Pionierem fizyki kwantowej był Max Planck, który w 1900 r. przyjął, że energia fal elektromagnetycznych emitowanych przez ciało doskonale czarne jest skwantowana i proporcjonalna do częstotliwości fali.
\index[persons]{Planck, Max}%
\index[persons]{de Broglie, Louis}%
W 1905 r. Albert Einstein wyjaśnił efekt fotoelektryczny, w 1913 r. Niels Bohr wyjaśnił skwantowanie poziomów energetycznych w atomie wodoru.
\index[persons]{Einstein, Albert}%
\index[persons]{Bohr, Niels}%
Rok 1922 przyniósł odkrycie Arthura Comptona: korpuskularny charakter fotonu, zaś dwa lata później francuski fizyk Louis de Broglie przyjął, że poruszający się elektron ma własności falowe.
\index[persons]{Compton, Arthur}%
\index[persons]{de Broglie, Louis}%
W 1925 r. Werner Heisenberg, Max Born i Pascual Jordan sformułowali macierzową reprezentację mechaniki kwantowej.
\index[persons]{Heisenberg, Werner}%
\index[persons]{Born, Max}%
\index[persons]{Jordan, Pascual}%
Rok później Erwin Schrödinger opublikował konkurencyjną teorię (mechanika falowa).
Obydwa opisy okazały się równoważne.

W 1927 r. Werner Heisenberg sformułował zasadę nieoznaczoności:

\begin{theorem}[zasada nieoznaczoności Heisenberga]
	Nie można wyznaczyć jednocześnie położenia i pędu cząstki z dowolną dokładnością:
	\begin{equation}
		\sigma_x \sigma_p \ge \frac{h}{4\pi}.
	\end{equation}
\end{theorem}

orbital [łac. orbita ‘koleina’, ‘droga’]: funkcja falowa $\phi$ opisująca stan jednego elektronu, zależna od współrzędnych określających jego położenie w atomie (orbital atomowy), cząsteczce (orbital molekularny, orbital cząsteczkowy) lub krysztale.

W 1927 niemiecki astrofizyk Albrech Unsöld stwierdził, że kwadrat funkcji falowej dla podpowłoki wypełnionej całkowicie lub w połowie jest sferycznie symetryczny.