% masę elektronu wyznaczył Millikan

\section{Budowa atomu}
% Jones, Atkins: 1, 7, 9
Nad teorią dotyczącą budowy materii pracowano już w starożytności.
Greccy filozofowie Leucyp z Miletu oraz jego uczeń Demokryt z Abdery sądzili, że otaczające ich substancje są zbudowane z małych niepodzielnych drobin (z greckiego \emph{atomos} -- niepodzielny).
\index[persons]{Leucyp z Miletu}%
\index[persons]{Demokryt z Abdery}%
Ich pomysły rozwijali potem Epikur oraz Lukrecjusz, ale niepoparte doświadczeniami pozostały zapomniane aż do wczesnego średniowiecza.
\index[persons]{Epikur}%
\index[persons]{Lukrecjusz}%
Atomizm kojarzono z epikureizmem, który był sprzeczny z nauczaniem chrześcijańskim, więc większość europejskich filozofów nie akceptowała istnienia atomów.
Francuski duchowny Pierre Gassendi zaproponował pewne zmiany w tych koncepcjach i jako pierwszy użył terminu ,,cząsteczka''.
% molecule
\index[persons]{Gassendi, Pierre}%
Obrońcami atomizmu byli angielski chemik Robert Boyle oraz angielski fizyk Izaak Newton tak, że pod koniec XVII wieku pogląd ten był już akceptowany przez część społeczności naukowej.
\index[persons]{Boyle, Robert}%
\index[persons]{Newton, Izaak}%

Pod koniec XVIII wieku znano już prawo zachowania masy:
\begin{theorem}[prawo zachowania masy]
	Suma mas spoczynkowych ciał biorących udział w procesie nie zmienia się.
\end{theorem}
(sformułowane niezależnie od siebie przez rosyjskiego chemika Michaiła Łomonosowa w 1756 r. i francuskiego chemika Antoine de Lavoisier w 1785 r.) oraz prawo stosunków stałych albo stałości składu:
\index[persons]{Łomonosow, Michaił}%
\index[persons]{Lavoisier, Antoine}%
\begin{theorem}[prawo stosunków stałych]
	Pierwiastki chemiczne łączą się ze sobą w określony związek chemiczny zawsze w tym samym stosunku wagowym.
\end{theorem}
(podane po raz pierwszy około 1799 r. przez francuskiego chemika Josepha Prousta, potwierdzone przez belgijskiego chemika Jeana Stasa).
\index[persons]{Proust, Joseph}%
\index[persons]{Stas, Jean}%
John Dalton zaproponował jeszcze prawo stosunków wielokrotnych:
\index[persons]{Dalton, John}%

\begin{tcolorbox}[title={Do zrobienia poźniej}]
Oprócz związków spełniających prawo Prousta (tzw. daltonidy) istnieją związki nie spełniające go (tzw. bertolidy).
\end{tcolorbox}

\begin{theorem}[prawo stosunków wielokrotnych]
Jeżeli dwa pierwiastki A i B tworzą ze sobą więcej niż jeden związek, to masy pierwiastka A przypadające na taką samą masę pierwiastka B mają się do siebie jak niewielkie liczby całkowite.
\end{theorem}

Ten sam Dalton ogłosił w 1803 r. założenia hipotezy atomistycznej, które tłumaczyły doświadczalne prawa chemiczne.
Założeniami tymi były:
\begin{compactitem}
	\item materia zbudowana jest z atomów, 
	\item atomy jednego pierwiastka są identyczne i mają kształt kuli,
	\item atomy są niezmienne i niepodzelne,
	\item związki chemiczne powstają przez łączenie się atomów.
\end{compactitem}

% \subsection{Avogadro}
% Usterkę w teorii Daltona poprawił Amedeo Avogadro w 1811 roku.
% Zasugerował on, że równe objętości dowolnych dwóch gazów w tej samej temperaturze i pod tym samym ciśnieniem zawierają równe ilości cząsteczek tych gazów.
% Prawo to pozwoliło wywnioskować, że wiele gazów ma dwuatomowe cząsteczki. % , na przykład: dwa litry wodoru reagują z litrem tlenu i dają dwa litry pary wodnej

\textbf{Ruchy Browna}.
W 1827 r. szkocki botanik Robert Brown zauważył bezładne ruchy po zawikłanych torach pyłków kwiatowych unoszących się na wodzie.
Ruch ten nie słabnie w czasie i staje się bardziej intensywny wraz ze wzrostem temperatury wody.
\index[persons]{Brown, Robert}%
W 1881 r. polski fizyk Łukasz Bodaszewski zaobserwował i opisał podobne zjawisko w gazach.
\index[persons]{Bodaszewski, Łukasz}%
Pierwsze teoretyczne wyjaśnienie ruchów Browna podali niezależnie Albert Einstein w 1905 r. i Marian Smoluchowski w 1906 roku: cząsteczki wody nieustannie uderzają o pyłki i powodują ich ruch.
(Matematyczny opis ruchów Browna był jedną z przyczyn, dla których powstała teoria procesów stochastycznych).
\index[persons]{Einstein, Albert}%
\index[persons]{Smoluchowski, Marian}%
Wyniki Einsteina potwierdził w 1908 r. na drodze doświadczalnej francuski fizyk Jean Perrin.
\index[persons]{Perrin, Jean}%

\textbf{Odkrycie elektronu, model ciasta z rodzynkami Thomsona}.
Atomy uważano za najmniejsze porcje materii aż do 1897 r., kiedy Joseph Thomson, John Townsend, Harold Wilson odkryli ,,korpuskuły'' podczas swoich doświadczeń nad promieniowaniem katodowym w rurze Crookesa-Hittorfa; ostatecznie przyjęła się nazwa zaproponowana w 1891 r. przez irlandzkiego fizyka George'a Stoneya: ,,\emph{elektrony}''.
\index[persons]{Thomson, Joseph}%
\index[persons]{Townsen, John}%
\index[persons]{Wilson, Harold}%
\index{rura Crookesa-Hittorfa}%
% https://en.wikipedia.org/wiki/Crookes_tube
Thomson rozstrzygnął przez zastosowanie komory Wilsona, że masa elektronu jest 1800 razy mniejsza niż masa cząsteczki wodoru.
\index{komora Wilsona}%
Aby wytłumaczyć, czemu ładunek całego atomu jest zerowy, podał \emph{model ciasta z rodzynkami}: elektrony to punktowe ujemne ładunki w ciągłym przestrzennie ładunku dodatnim.

\textbf{Odkrycie jądra, model planetarny Rutherforda}.
Hans Geiger i Ernest Marsden pod nadzorem Ernesta Rutherforda w latach 1908-1913 bombardowali cząstkami $\alpha$ cienką folię ze złota .
\index[persons]{Geiger, Hans}%
\index[persons]{Marsden, Ernest}%
\index[persons]{Rutherford, Ernest}
Gdyby model Thomsona był prawdziwy, cząstki $\alpha$ powinny przenikać przez folię, ewentualnie z lekkim odchyleniem toru.
Ale uczeni zauważyli duży rozrzut cząstek na fluoresencyjnym ekranie wokół folii, niektóre z nich odskakiwały w tył.
Doszli do wniosku, że niemal cała masa i dodatni ładunek skupione są w jądrze atomowym o bardzo małych rozmiarach; jednocześnie obalając model Thomsona.
Według Rutherforda pomysł planetarnego modelu pochodził od Hantaro Nagaoki (,,pierścienie wokół Saturna'').
\index[persons]{Hantaro Nagaoka}

\textbf{Model Bohra}.
Planetarny model atomu miał dwie poważne wady: po pierwsze elektrony powinny emitować fale elektromagnetyczne zgodnie z wzorem\footnote{$P = q^2a^2/(6\pi\varepsilon_0 c^3$} Larmora, stale tracić energię i wirować w kierunku jądra, zderzając się z nim w czasie około $10^{-6} s$.
Po drugie, model ten nie tłumaczył wysokich pików na widmach emisyjnych i absorbcyjnych atomów.
% Kwantowy model A. Haasa z 1910 r. oraz J. Nicholsona z 1912 roku.
By wyjaśnić te sprzeczności duński fizyk Niels Bohr zaproponował w 1913 r. teorię budowy atomu, gdzie elektron krąży wokół jądra atomowego po orbitach stacjonarnych o ściśle określonej energii.
\index[persons]{Bohr, Niels}%
Orbitalny moment pędu jest skwantowany (jest całkowitą wielokrotnością h kreślonego).
Elektron przechodząc między orbitami emituje lub pochłania foton i nie może spaść na jądro atomowe.
Model Bohra nie był doskonały: przewidywał jedynie linie widmowe wodoru, ale nie radził sobie z atomami o większej liczbie elektronów.
Wraz z rozwojem spektrografii zaobserwowano dodatkowe linie widmowe wodoru, których ten model nie tłumaczył.
W 1916 r. Arnold Sommerfeld dokonał poprawki modelu przez dodanie eliptycznych orbit, ale to uczyniło go trudnym w użyciu i nadal nie tłumaczyło innych pierwiastków.
\index[persons]{Sommerfeld, Arnold}%

\textbf{Odkrycie izotopów}.
W 1913 r. brytyjski chemik Frederick Soddy odkrył, że radioaktywne pierwiastki mogą mieć wiele odmian o różnych masach atomowych, ale tych samych własnościach chemicznych.
Zgodnie z sugestią Margaret Todd nazwał te odmiany \emph{izotopami}.
\index[persons]{Soddy, Frederick}%
\index[persons]{Todd, Margaret}%
Thomson odkrył z Francisem Astonem w 1913 r. na przykładzie neonu, że nieradioaktywne pierwiastki też mogą mieć wiele izotopów.
\index[persons]{Thomson, Joseph}%
\index[persons]{Aston, Francis}%

% Soddy przewidział także końcowe produkty rozpadu szeregu uranowego i torowego (są to izotopy ołowiu o masach atomowych 206 i 208).
% W 1917 odkrył izotop protaktynu 231Pa (niezależnie od Soddy’ego odkryli go także O. Hahn i L. Meitner).
% Soddy i Rutherford wprowadzili pojęcie „czasu połowicznego rozpadu” pierwiastków promieniotwórczych.

\textbf{Odkrycie protonu}.
% TODO: The concept of a hydrogen-like particle as a constituent of other atoms was developed over a long period. As early as 1815, William Prout proposed that all atoms are composed of hydrogen atoms (which he called "protyles"), based on a simplistic interpretation of early values of atomic weights (see Prout's hypothesis), which was disproved when more accurate values were measured.[16]: 39–42 
% In 1886, Eugen Goldstein discovered canal rays (also known as anode rays) and showed that they were positively charged particles (ions) produced from gases. However, since particles from different gases had different values of charge-to-mass ratio (e/m), they could not be identified with a single particle, unlike the negative electrons discovered by J. J. Thomson. Wilhelm Wien in 1898 identified the hydrogen ion as the particle with the highest charge-to-mass ratio in ionized gases.[17]
% Following the discovery of the atomic nucleus by Ernest Rutherford in 1911, Antonius van den Broek proposed that the place of each element in the periodic table (its atomic number) is equal to its nuclear charge. This was confirmed experimentally by Henry Moseley in 1913 using X-ray spectra.


W 1917 r. Rutherford pokazał, że jądro wodoru jest obecne w jądrach innych pierwiastków.
Zauważył najpierw, że kiedy strzelał cząsteczkami $\alpha$ w powietrzu (złożonym głównie z azotu), jednym z produktów były jądra wodoru, a potem że to samo doświadczenie przeprowadzone w czystym azocie daje mocniejszy efekt.
W 1919 r. założył, że promieniowanie $\alpha$ wybija proton z azotu, zamieniając go w węgiel, ale po 1925 r. zrozumiał, że równocześnie cząstka $\alpha$ była pochłaniania, zaś jądra wodoru emitowane, dając atomy tlenu, nie węgla: \ce{^{14}N + {\alpha} -> ^{17}O + p}.

Rutherford wiedział, że dodatni ładunek dowolnego atomu jest całkowitą krotnością jąder wodoru, ale przypuszczał więcej: że jądro wodoru jest pojedynczą cząstką, podstawowym składnikiem wszystkich innych jąder i nazwał je protonem.
Dalsze eksperymenty pokazały, że masa jądra przekracza masę protonów, spekulował, że różnica ta bierze się z nieznanych dotąd obojętnych elektrycznie cząstek, które nazwał neutronami.

\textbf{Odkrycie neutronu}.
W 1930 r. niemiecki fizyk Walther Bothe oraz jego student Herbert Becker bombardowali beryl cząstkami $\alpha$ i zauważyli, że emituje on promieniowanie, które przechodzi nawet przez 20-centymetrową ścianę z ołowiu.
Podobne eksperymenty przeprowadzali Frederic Joliot i Irena Joliot-Curie oraz James Chadwick, który w pewnej odległości od tarczy umieścił wosk parafinowy.
Energia promieniowania Roentgena wystarczała do uwolnienia elektronów z wosku, ale nie protonów.
Dlatego Chadwick stwierdził, że protony z wosku wybijane były przez inne promieniowanie obojętnych cząstek o masie zbliżonej do masy protonu, \emph{neutrony}.
W tym samym roku Dmitryj Iwanienko zasugerował, że neutrony i protony są odpowiedzialne za masę jądra.
Umożliwiło to obliczenie energii wiązań poszczególnych jąder.

\textbf{Kwantowe modele atomu}.
% TODO!


\subsection{Proton}
Proton to trwała cząstka (z grupy barionów) o ładunku +1e, masie spoczynkowej równej $\approx 1 u$ oraz promieniem około $0,833 fm \pm 0,010 fm$ (zależy, kto mierzył XD)
Zbudowany z trzech kwarków: dwóch górnych i jednego dolnego.

1.007276466621(53) Da

Protony i neutrony => nukleony?

% GREKA TODO: The word proton is Greek for "first", and this name was given to the hydrogen nucleus by Ernest Rutherford in 1920.





ELektron to cząstka o ładunku $-1e$, masie równej około $1/1836$ masy protonu
masa: % [1822.8884845(14)]-1 u
lepton

theorized ... discovered ... promien ok. 10 do -18..-22




Neutron:
2 down quarks
1 up quark
theorized, discovereed, masa, promień