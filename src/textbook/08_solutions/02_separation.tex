
\section{Metody rozdzielania}
Inne:
filtracja, 
ekstrakcja,
krystalizacja, 
chromatografia, 
elektroforeza, 
wymiana jonowa, 
flotacja, 
osmoza, 
rozdział grawitacyjny (opiszę je później).
% TODO

\subsection{Metody rozdzielania mieszanin jednorodnych}
\begin{compactitem}
\item roztwór właściwy substancji stałej w cieczy: \begin{compactitem}
	\item krystalizacja,
	\item odparowanie rozpuszczalnika,
	\item destylacja. % i rektyfikacja?
\end{compactitem}
\item roztwór właściwy cieczy w cieczy: \begin{compactitem}
	\item destylacja frakcyjna.
\end{compactitem}
\item roztwór właściwy w gazie lub cieczy: \begin{compactitem}
	\item adsorpcja.
\end{compactitem}
\end{compactitem}

\subsection{Metody rozdzielania mieszanin niejednorodnych}
\begin{compactitem}
\item mieszanina substancji stałych o różnym stopniu rozdrobnienia: \begin{compactitem}
	\item przesiewanie.
\end{compactitem}
\item roztwór właściwy cieczy w cieczy: \begin{compactitem}
	\item rozdzielanie magnesem.
\end{compactitem}
\item roztwór właściwy w gazie lub cieczy: \begin{compactitem}
	\item zastosowanie rozdzielacza.
\end{compactitem}
\item roztwór właściwy w gazie lub cieczy: \begin{compactitem}
	\item sedymentacja,
	\item dekantacja,
	\item sączenie.
\end{compactitem}
\end{compactitem}