
\section{Osmoza}

% SŁOWNICZEK
% Semipermeable membrane : membrana półprzepuszczalna

\important{Osmoza} to dyfuzja (samorzutny przepływ) rozpuszczalnika przez membranę rozdzielającą dwa roztwory o różnych stężeniach.
Na zasadzie osmozy działają aparaty do dializy nerek, gdzie po dwóch stronach półprzepuszczalnej błony znajdują się krew do oczyszczenia oraz sterylny roztwór.

\begin{etymology_box}
Słowo osmoza (\textgreek{ὠσμός}, osmos: ,,pchnięcie'') powstało przez analogię do endosmozy (\textgreek{ἔνδον}, endon: ,,wewnątrz''?) i eksosmozy (\textgreek{ἔξω}, ekso: ,,zewnątrz''?), które wymyślił francuski lekarz Henri Dutrochet.
\end{etymology_box}
\index[persons]{Dutrochet, Henri}%

Jako pierwszy zjawisko osmozy opisał francuski fizyk Jean-Antoine Nollet w 1748 roku: wypełniony alkoholem odtłuszczony świński pęcherz zanurzył w wodzie.
\index[persons]{Nollet, Jean-Antoine}%
Po upływie godzin do pęcherza dostało się tyle wody, że ten wybrzuszył się; przekłucie go doprowadziło do wystrzelenia wody.

W 1864 roku pruski chemik Moritz Traube stworzył sztuczną  \important{membranę półprzepuszczalną} umieszczając krople kleju w kwasie taninowym, które potem rosły przez infuzję wody.
\index{kwas taninowy}
\index[persons]{Traube, Moritz}%
Oprócz kleju eksperymentował także z grynszpanem (hydroksyoctanem miedzi (II), substancją barwiącą do farb) albo żelazocyjankiem potasu i chlorkiem miedzi.
\index{hydroksyoctan miedzi}%
\index{żelazocyjanek potasu}%
\index{chlorek potasu}%

Roztwór, z którego ubywa rozpuszczalnika, nazywamy \important{hipotonicznym}; ten w którym go przybywa -- \important{hipertonicznym}.
Jeżeli roztwory znajdują się w równowadze osmotycznej, to obydwa nazywa się \important{izotonicznymi} (względem siebie).

\textbf{Do zrobienia:
Ciśnienie osmotyczne.
Osmometria.
Odwrócona osmoza.
Równanie van't Hoffa.
}