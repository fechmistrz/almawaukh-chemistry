% Czarny zeszyt: strony 22, 23
Mieszaniny jednorodne (homogeniczne) nazywamy roztworami.
\index{mieszanina jednorodna}%
\index{roztwór}%
Wzrokowe określenie składu roztworu jest niemożliwy, tak jest na przykład w stopach metali, benzynie, occie czy powietrzu.
Podział na mieszaniny jednorodne i niejednorodne jest umowny!
Piasek z wodą albo opiłki metalu zmieszane z cukrem stanowią przykłady mieszanin niejednorodnych, nie o nich jest ten rozdział.

\begin{center}
\begin{tabular}{c|ccc}
\textbf{\makecell{co $\to$\\w czym}} & \textbf{gaz} & \textbf{ciecz} & \textbf{ciało stałe} \\ \hline
\textbf{gaz} & --- & \makecell{aerozol \emph{(mgła, chmury,}\\\emph{lakier w sprayu)}} & \makecell{aerozol \emph{(dym, kurz,}\\\emph{sadza w płomieniu)}} \\
\textbf{ciecz} & \makecell{piana \emph{(bita śmietana,}\\\emph{pianka do golenia)}} & \makecell{emulsja \emph{(mleko, }\\\emph{majonez, niektóre farby)}} & zol \emph{(krew, błoto)} \\
\textbf{\makecell{ciało\\stałe}} & \makecell{piana stała\\\emph{(pumeks)}} & \makecell{emulsja stała \emph{(kwarc mleczny,}\\\emph{bitumiczne nawierzchnie dróg)}} & \makecell{zol stały\\\emph{(rubin, ametyst)} }
\end{tabular}
\end{center}

W roztworach wyróżnia się \important{rozpuszczalnik} (najbardziej znanym jest woda) oraz \important{substancję rozpuszczaną}.
\index{rozpuszczalnik}%
\index{substancja rozpuszczona}%
% WIKI-pl: Większość rozpuszczalników to związki chemiczne o małej lepkości i stosunkowo niskiej temperaturze wrzenia. Mała lepkość powoduje, że mogą one dość łatwo penetrować rozpuszczaną substancję, zaś niska temperatura wrzenia umożliwia ich oddestylowywanie i parowanie.
Miarą ilości substancji rozpuszczonej jest stężenie.
Można wyrazić je na kilka sposobów: \important{molowość} (mole substancji rozpuszczonej na litr roztworu), \important{molalność} (mole substancji rozpuszczonej na kilogram rozpuszczalnika), procent objętościowy oraz masowy i inne.
{
\color{red}
Popularną niegdyś metodą, choć stosowaną czasem także obecnie, był pomiar gęstości roztworu za pomocą areometru Baumégo wyskalowanego w stopniach Baumégo (°Bé). W tych jednostkach podawano stężenie m.in. kwasu siarkowego[18], kwasu azotowego[19] i roztworów glukozy[20]. Dla wodnych roztworów etanolu zastosowanie mają stopnie Richtera (°R), odpowiadające jednostce „g/100 g” (procentowi masowemu), oraz stopnie Trallesa (°Tr) i stopnie Gay-Lussaca (°GL), które odpowiadają jednostce „ml/100 ml” (procentowi objętościowemu)[21][22][23]. Analogicznie dla roztworów sacharozy stosuje się stopnie Ballinga (°Blg) i stopnie Brixa (°Bx) odpowiadające jednostce „g/100 g”[24].
}


Maksymalne stężenie jakie można uzyskać w danych warunkach nazywamy \important{rozpuszczalnością}, a roztwór o takim stężeniu -- \important{roztworem nasyconym}.

{\color{red}
Roztwór przesycony to roztwór o stężeniu większym od stężenia roztworu nasyconego w danej temperaturze. Roztwory przesycone są przykładami substancji w stanie termodynamicznym niestabilnym metatrwałym.


Miód – przykład roztworu przesyconego
Roztwór przesycony można otrzymać przez uzyskanie roztworu nasyconego w temperaturze wyższej, pozbawienie go pozostałej stałej substancji rozpuszczanej (tak żeby nie było zarodków krystalizacji), a następnie ostrożne oziębianie tego roztworu. Roztwór przesycony jest termodynamicznie nietrwały. Wprowadzenie zaburzenia (np. wstrząs, kurz) może spowodować krystalizację nadmiaru substancji rozpuszczonej.

Znanym z życia codziennego przykładem roztworu przesyconego jest miód, w którym krystalizacja glukozy w temperaturze pokojowej może trwać kilka miesięcy, a nawet kilka lat.

Efekt uwalniania ciepła podczas szybkiej krystalizacji z roztworu przesyconego znajduje zastosowanie np. w ogrzewaczach dłoni.
Można je osiągnąć poprzez ochłodzenie roztworu nasyconego, przez odparowanie rozpuszczalnika, dodatek substancji wysalającej lub w wyniku reakcji chemicznej między dwiema jednorodnymi fazami.}

{\color{red}
Ogólną zasadą jest, że podobne rozpuszcza się w podobnym. Na przykład związek chemiczny, który posiada kwaśny atom wodoru albo dużo grup hydroksylowych będzie się chętniej rozpuszczał w rozpuszczalnikach protonowych niż aprotonowych. Podobnie, związek, który sam posiada duży moment dipolowy, będzie się chętniej rozpuszczał w rozpuszczalnikach polarnych niż apolarnych. Związki zawierające grupy aromatyczne będą się chętnie rozpuszczać w arenach.
}

Ze względu na wielkość cząstek substancji rozpuszczonej w ciekłym rozpuszczalniku wyróżnia się:
\begin{compactitem}
\item \important{roztwory właściwe} (średnica mniejsza niż \SI{e-9}{\metre}, np. roztwór soli w wodzie)
\item \important{koloidy} (średnica \SI{e-9}{\metre} do \SI{e-7}{\metre}, np. mleko) oraz 
\item \important{zawiesiny} (średnica większa niż \SI{e-7}{\metre}, np. piasek w wodzie).
\end{compactitem}
\index{koloid}%
\index{zawiesina}%
\index{roztwór właściwy}%
Czasami koloidem określa się substancję rozproszoną, wtedy cały roztwór jest zawiesiną koloidalną. % colloid, colloidal suspension

\important{Emulsja} to mieszanina dwóch nierozpuszczających się w sobie cieczy, z których jedna jest rozproszona w drugiej w postaci bardzo małych kropelek.
Powstają przez wstrząsanie lub wkraplanie jednej cieczy do drugiej przy jednoczesnym mieszaniu.
Do trwałego połączenia składników potrzebne są \important{emulgatory}, przykładem którego jest mydło.

Woda jest dobrym rozpuszczalnikiem substancji o budowie polarnej i jonowej.

\textbf{TODO: roztwarzanie}





% https://pl.wikipedia.org/wiki/Stężenie
% https://pl.wikipedia.org/wiki/Roztwór_przesycony