
\section{Rozpuszczalność}
\important{Rozpuszczalność} (maksymalna masa substancji rozpuszczonej w 100 g rozpuszczalnika) zależy od temperatury, a dla gazów -- także ciśnienia.
Wraz ze wzrostem temperatury maleje dla gazów i rośnie dla większości soli.

\important{Roztwór nasycony} to taki, który zawiera substancję rozpuszczoną w ilości równej rozpuszczalności.
Schładzanie roztworu nasyconego powoduje wytrącenie z niego kryształów substancji rozpuszczonej (\important{krystalizacja}).

\textbf{TODO: roztwór przesycony, zarodek krystalizacji}

Stężenie substancji w mieszaninie podaje się przede wszystkim na dwa sposoby: \important{stężenie procentowe} to stosunek masy substancji rozpuszczonej do masy całego roztworu, \important{stężenie molowe} to stosunek liczby moli substancji rozpuszczonej do objętości całego roztworu.