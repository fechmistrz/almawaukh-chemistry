\section{Zol jako przykład koloidu}
Koloidy można otrzymywać na dwa sposoby: za pomocą \important{dyspersji} polegającej na rozdrabnianiu cząsteczek przekraczających 500 nm.  W tym celu wykorzystać można takie metody jak rozdrabnianie mechaniczne, metodę Brediga, peptyzację, rozpylanie za pomocą ultradźwięków lub termiczne.
Metodą odwrotną jest \important{kondensacja}. Polega na łączeniu cząsteczek lub jonów w większe zespoły tak, by ich wymiar osiągnął wymiary charakterystyczne dla cząsteczek koloidalnych. Wykorzystuje się do tego m.in.: zmniejszenie rozpuszczalności, redukcję, metodę zarodnikową, utlenianie oraz polimeryzację.

Wiązka światła przechodząca przez koloid rozprasza się z wytworzeniem charakterystycznego stożka, zjawisko to nazywamy \important{efektem Tyndalla}, ponieważ irlandzki fizyk John Tyndall zbadał je w 1859 r.
\index{efekt Tyndalla}%
\index[persons]{Tyndall, John}%

Pod wpływem soli zole ulegają \important{koagulacji} (cząstki fazy rozproszonej łączą się w większe skupiska), powstają galaretowate osady nazywane \important{żelami}.
Koagulację powodują też inne czynniki, np. dodanie alkoholu, podwyższenie temperatury albo wstrząsanie.
Jest to proces odwracalny tylko dla kolodiów liofilowych, wystarczy dodać wody (mówimy wtedy o \important{peptyzacji}).
Nieodwracalny proces koagulacji białka to \important{denaturacja}, w wyniku której następuje zmiana struktury i utrata własności funkcjonalnych.

\textbf{TODO: dializa, dializator, sztuczna nerka}